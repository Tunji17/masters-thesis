%%%%%%%%%%%%%%%%%%%%%%%%%%%%%%%%%%%%%%%%%
% Masters/Doctoral Thesis 
% LaTeX Template
% Version 2.5 (27/8/17)
%
% This template was downloaded from:
% http://www.LaTeXTemplates.com
%
% Version 2.x major modifications by:
% Vel (vel@latextemplates.com)
%
% This template is based on a template by:
% Steve Gunn (http://users.ecs.soton.ac.uk/srg/softwaretools/document/templates/)
% Sunil Patel (http://www.sunilpatel.co.uk/thesis-template/)
%
% Template license:
% CC BY-NC-SA 3.0 (http://creativecommons.org/licenses/by-nc-sa/3.0/)
%
%%%%%%%%%%%%%%%%%%%%%%%%%%%%%%%%%%%%%%%%%

%----------------------------------------------------------------------------------------
%	PACKAGES AND OTHER DOCUMENT CONFIGURATIONS
%----------------------------------------------------------------------------------------

\documentclass[
11pt, % The default document font size, options: 10pt, 11pt, 12pt
%oneside, % Two side (alternating margins) for binding by default, uncomment to switch to one side
english, % ngerman for German
singlespacing, % Single line spacing, alternatives: onehalfspacing or doublespacing
%draft, % Uncomment to enable draft mode (no pictures, no links, overfull hboxes indicated)
%nolistspacing, % If the document is onehalfspacing or doublespacing, uncomment this to set spacing in lists to single
%liststotoc, % Uncomment to add the list of figures/tables/etc to the table of contents
%toctotoc, % Uncomment to add the main table of contents to the table of contents
%parskip, % Uncomment to add space between paragraphs
%nohyperref, % Uncomment to not load the hyperref package
headsepline, % Uncomment to get a line under the header
%chapterinoneline, % Uncomment to place the chapter title next to the number on one line
%consistentlayout, % Uncomment to change the layout of the declaration, abstract and acknowledgements pages to match the default layout
]{MastersDoctoralThesis} % The class file specifying the document structure

\usepackage[utf8]{inputenc} % Required for inputting international characters
\usepackage[T1]{fontenc} % Output font encoding for international characters

\usepackage{mathpazo} % Use the Palatino font by default

\usepackage[backend=bibtex,style=authoryear,natbib=true]{biblatex} % Use the bibtex backend with the authoryear citation style (which resembles APA)

\addbibresource{bibliography.bib} % The filename of the bibliography

\usepackage[autostyle=true]{csquotes} % Required to generate language-dependent quotes in the bibliography

\usepackage{tikz} % TikZ for drawing diagrams and figures
\usetikzlibrary{arrows.meta, positioning, shapes.geometric}

%----------------------------------------------------------------------------------------
%	MARGIN SETTINGS
%----------------------------------------------------------------------------------------

\geometry{
	paper=a4paper, % Change to letterpaper for US letter
	inner=2.5cm, % Inner margin
	outer=3.8cm, % Outer margin
	bindingoffset=.5cm, % Binding offset
	top=1.5cm, % Top margin
	bottom=1.5cm, % Bottom margin
	%showframe, % Uncomment to show how the type block is set on the page
}

%----------------------------------------------------------------------------------------
%	THESIS INFORMATION
%----------------------------------------------------------------------------------------

\thesistitle{Transforming Unstructured Biomedical Text into Actionable Knowledge Graphs: A Multi-Model Approach for Healthcare Information Extraction} % Your thesis title, this is used in the title and abstract, print it elsewhere with \ttitle
\supervisor{Maziyar \textsc{PANAHI}} % Your supervisor's name, this is used in the title page, print it elsewhere with \supname
\degree{Master of Science} % Your degree name, this is used in the title page and abstract, print it elsewhere with \degreename
\author{Oyetunji Daniel \textsc{Abioye}} % Your name, this is used in the title page and abstract, print it elsewhere with \authorname
\addresses{} % Your address, this is not currently used anywhere in the template, print it elsewhere with \addressname

\subject{Computer Science} % Your subject area, this is not currently used anywhere in the template, print it elsewhere with \subjectname
\keywords{} % Keywords for your thesis, this is not currently used anywhere in the template, print it elsewhere with \keywordnames
\university{\href{https://www.univ-lorraine.fr/en/univ-lorraine}{Université de lorraine}} % Your university's name and URL, this is used in the title page and abstract, print it elsewhere with \univname
\department{\href{https://idmc.univ-lorraine.fr}{L'Institut des Sciences du Digital Management \& Cognition}} % Your department's name and URL, this is used in the title page and abstract, print it elsewhere with \deptname
\group{\href{https://idmc.univ-lorraine.fr}{IDMC}} % Your research group's name and URL, this is used in the title page, print it elsewhere with \groupname
\faculty{\href{https://idmc.univ-lorraine.fr}{IDMC}} % Your faculty's name and URL, this is used in the title page and abstract, print it elsewhere with \facname

\AtBeginDocument{
\hypersetup{pdftitle=\ttitle} % Set the PDF's title to your title
\hypersetup{pdfauthor=\authorname} % Set the PDF's author to your name
\hypersetup{pdfkeywords=\keywordnames} % Set the PDF's keywords to your keywords
}

\begin{document}

\frontmatter % Use roman page numbering style (i, ii, iii, iv...) for the pre-content pages

\pagestyle{plain} % Default to the plain heading style until the thesis style is called for the body content

%----------------------------------------------------------------------------------------
%	TITLE PAGE
%----------------------------------------------------------------------------------------

\begin{titlepage}
\begin{center}

\vspace*{.06\textheight}
{\scshape\LARGE \univname\par}\vspace{1.5cm} % University name
\textsc{\Large Master's Thesis}\\[0.5cm] % Thesis type

\HRule \\[0.4cm] % Horizontal line
{\huge \bfseries \ttitle\par}\vspace{0.4cm} % Thesis title
\HRule \\[1.5cm] % Horizontal line
 
\begin{minipage}[t]{0.4\textwidth}
\begin{flushleft} \large
\emph{Author:}\\
\href{https://www.oyetunji.com}{\authorname} % Author name - remove the \href bracket to remove the link
\end{flushleft}
\end{minipage}
\begin{minipage}[t]{0.4\textwidth}
\begin{flushright} \large
\emph{Supervisor:} \\
\href{https://iscpif.fr}{\supname} \\ % Supervisor name
\end{flushright}
\end{minipage}\\[3cm]

\vfill

\large \textit{A thesis submitted in fulfillment of the requirements\\ for the degree of \degreename}\\[0.3cm] % University requirement text
\textit{in the}\\[0.4cm]
\groupname\\\deptname\\[2cm] % Research group name and department name
 
\vfill

{\large \today}\\[4cm] % Date
%\includegraphics{Logo} % University/department logo - uncomment to place it
 
\vfill
\end{center}
\end{titlepage}

%----------------------------------------------------------------------------------------
%	DECLARATION PAGE
%----------------------------------------------------------------------------------------

\begin{declaration}
\addchaptertocentry{\authorshipname} % Add the declaration to the table of contents
\noindent I, \authorname, declare that this thesis titled, \enquote{\ttitle} and the work presented in it are my own. I confirm that:

\begin{itemize} 
\item This work was done wholly or mainly while in candidature for a research degree at \univname.
\item Where any part of this thesis has previously been submitted for a degree or any other qualification at this University or any other institution, this has been clearly stated.
\item Where I have consulted the published work of others, this is always clearly attributed.
\item Where I have quoted from the work of others, the source is always given. With the exception of such quotations, this thesis is entirely my own work.
\item I have acknowledged all main sources of help.
\item Where the thesis is based on work done by myself jointly with others, I have made clear exactly what was done by others and what I have contributed myself.\\
\end{itemize}
 
\noindent Signed:\\
\rule[0.5em]{25em}{0.5pt} % This prints a line for the signature
 
\noindent Date:\\
\rule[0.5em]{25em}{0.5pt} % This prints a line to write the date
\end{declaration}

\cleardoublepage

%----------------------------------------------------------------------------------------
%	ABSTRACT PAGE
%----------------------------------------------------------------------------------------

\begin{abstract}
\addchaptertocentry{\abstractname} % Add the abstract to the table of contents
The exponential growth of biomedical text data presents significant challenges and opportunities for extracting actionable knowledge that can advance healthcare and biomedical research. Unstructured biomedical documents, including research abstracts and clinical notes, contain vast amounts of valuable information not readily accessible through traditional structured databases. This thesis proposes a comprehensive, multi-model pipeline designed to transform unstructured biomedical text into structured, actionable knowledge graphs.

Leveraging advanced natural language processing (NLP) techniques, the pipeline integrates specialized models for named entity recognition (GLiNER), entity linking to standardized medical knowledge bases (UMLS via SciSpacy), and relationship extraction using large language models (LLMs). Specifically, domain-specific models like MedGemma are compared against general-purpose models such as Gemma to evaluate their effectiveness in accurately extracting biomedical relationships. The pipeline's modular design ensures efficient processing, robust entity linking, and precise relationship extraction, addressing critical challenges such as synonymy, ambiguity, and the complexity of biomedical language.

The extracted data populates a knowledge graph constructed using Neo4j, allowing efficient storage, querying, and retrieval of complex biomedical relationships. The graph schema and query generation processes are carefully designed to ensure data integrity and scalability. Comprehensive evaluations using the BioRED dataset demonstrate the pipeline's high accuracy in entity recognition and relationship extraction, underscoring its utility in real-world clinical and research applications.

Overall, this thesis contributes to biomedical NLP by delivering an integrated solution that systematically converts unstructured biomedical literature into actionable, structured knowledge, facilitating accelerated research discovery and enhanced clinical decision-making.
\end{abstract}

%----------------------------------------------------------------------------------------
%	ACKNOWLEDGEMENTS
%----------------------------------------------------------------------------------------

\begin{acknowledgements}
\addchaptertocentry{\acknowledgementname} % Add the acknowledgements to the table of contents
First and foremost, I would like to express my heartfelt gratitude to my supervisor, Maziyar Panahi, whose invaluable guidance, continuous support, and expert advice significantly enriched my research journey. Their insightful feedback and encouragement have been crucial in shaping this thesis and my academic growth.

I extend my sincere appreciation to the faculty and staff at L'Institut des Sciences du Digital Management \& Cognition (IDMC) at Université de Lorraine, for providing an inspiring academic environment and the necessary resources that facilitated this research.

My deepest thanks to the community and contributors behind the biomedical datasets and open-source tools that formed the backbone of this project. Their dedication and open collaboration have profoundly advanced research possibilities in biomedical natural language processing and knowledge graph development.

Special thanks go to my peers and colleagues who offered their time, discussions, and valuable perspectives throughout the development of this work. Your camaraderie and intellectual engagement have greatly enriched my experience.

Lastly, I am eternally grateful to my family and friends for their unwavering support, encouragement, and patience throughout my academic journey. Their belief in me has been my greatest motivation.

Thank you all for being a vital part of this journey.
\end{acknowledgements}

%----------------------------------------------------------------------------------------
%	LIST OF CONTENTS/FIGURES/TABLES PAGES
%----------------------------------------------------------------------------------------

\tableofcontents % Prints the main table of contents

\listoffigures % Prints the list of figures

\listoftables % Prints the list of tables

%----------------------------------------------------------------------------------------
%	ABBREVIATIONS
%----------------------------------------------------------------------------------------

\begin{abbreviations}{ll} % Include a list of abbreviations (a table of two columns)

\textbf{AI} & \textbf{A}rtificial \textbf{I}ntelligence\\
\textbf{API} & \textbf{A}pplication \textbf{P}rogramming \textbf{I}nterface\\
\textbf{BERT} & \textbf{B}idirectional \textbf{E}ncoder \textbf{R}epresentations from \textbf{T}ransformers\\
\textbf{BiLSTM} & \textbf{Bi}directional \textbf{L}ong \textbf{S}hort-\textbf{T}erm \textbf{M}emory\\
\textbf{BioBERT} & \textbf{Bio}medical \textbf{BERT}\\
\textbf{BioRED} & \textbf{Bio}medical \textbf{R}elation \textbf{E}xtraction \textbf{D}ataset\\
\textbf{CNN} & \textbf{C}onvolutional \textbf{N}eural \textbf{N}etwork\\
\textbf{CPU} & \textbf{C}entral \textbf{P}rocessing \textbf{U}nit\\
\textbf{CRF} & \textbf{C}onditional \textbf{R}andom \textbf{F}ields\\
\textbf{CUI} & \textbf{C}oncept \textbf{U}nique \textbf{I}dentifier\\
\textbf{DNA} & \textbf{D}eoxyribonucleic \textbf{A}cid\\
\textbf{EHR} & \textbf{E}lectronic \textbf{H}ealth \textbf{R}ecords\\
\textbf{FN} & \textbf{F}alse \textbf{N}egatives\\
\textbf{FP} & \textbf{F}alse \textbf{P}ositives\\
\textbf{GLiNER} & \textbf{G}eneralist and \textbf{L}ightweight \textbf{N}amed \textbf{E}ntity \textbf{R}ecognition\\
\textbf{GPT} & \textbf{G}enerative \textbf{P}re-trained \textbf{T}ransformer\\
\textbf{GPU} & \textbf{G}raphics \textbf{P}rocessing \textbf{U}nit\\
\textbf{HIPAA} & \textbf{H}ealth \textbf{I}nsurance \textbf{P}ortability and \textbf{A}ccountability \textbf{A}ct\\
\textbf{ICD} & \textbf{I}nternational \textbf{C}lassification of \textbf{D}iseases\\
\textbf{IDMC} & \textbf{I}nstitut des \textbf{S}ciences du \textbf{D}igital \textbf{M}anagement \& \textbf{C}ognition\\
\textbf{IE} & \textbf{I}nformation \textbf{E}xtraction\\
\textbf{JSON} & \textbf{J}avaScript \textbf{O}bject \textbf{N}otation\\
\textbf{KB} & \textbf{K}nowledge \textbf{B}ase\\
\textbf{LLM} & \textbf{L}arge \textbf{L}anguage \textbf{M}odel\\
\textbf{LSTM} & \textbf{L}ong \textbf{S}hort-\textbf{T}erm \textbf{M}emory\\
\textbf{MedGemma} & \textbf{Med}ical \textbf{Gemma}\\
\textbf{MeSH} & \textbf{Me}dical \textbf{S}ubject \textbf{H}eadings\\
\textbf{ML} & \textbf{M}achine \textbf{L}earning\\
\textbf{NER} & \textbf{N}amed \textbf{E}ntity \textbf{R}ecognition\\
\textbf{NLP} & \textbf{N}atural \textbf{L}anguage \textbf{P}rocessing\\
\textbf{PMID} & \textbf{P}ubMed \textbf{ID}entifier\\
\textbf{PubMed} & \textbf{Pub}lic \textbf{Med}line\\
\textbf{PubMedBERT} & \textbf{PubMed} \textbf{BERT}\\
\textbf{RAM} & \textbf{R}andom \textbf{A}ccess \textbf{M}emory\\
\textbf{RNN} & \textbf{R}ecurrent \textbf{N}eural \textbf{N}etwork\\
\textbf{RxNorm} & \textbf{Rx}Norm (Drug Naming System)\\
\textbf{SciBERT} & \textbf{Sci}entific \textbf{BERT}\\
\textbf{SNOMED CT} & \textbf{S}ystematized \textbf{No}menclature of \textbf{Med}icine \textbf{C}linical \textbf{T}erms\\
\textbf{SOTA} & \textbf{S}tate-\textbf{o}f-\textbf{t}he-\textbf{A}rt\\
\textbf{SVM} & \textbf{S}upport \textbf{V}ector \textbf{M}achine\\
\textbf{TF-IDF} & \textbf{T}erm \textbf{F}requency-\textbf{I}nverse \textbf{D}ocument \textbf{F}requency\\
\textbf{TP} & \textbf{T}rue \textbf{P}ositives\\
\textbf{UMLS} & \textbf{U}nified \textbf{M}edical \textbf{L}anguage \textbf{S}ystem\\
\textbf{VRAM} & \textbf{V}ideo \textbf{RAM}\\

\end{abbreviations}

%----------------------------------------------------------------------------------------
%	DEDICATION
%----------------------------------------------------------------------------------------

\dedicatory{For/Dedicated to/To my\ldots} 

%----------------------------------------------------------------------------------------
%	THESIS CONTENT - CHAPTERS
%----------------------------------------------------------------------------------------

\mainmatter % Begin numeric (1,2,3...) page numbering

\pagestyle{thesis} % Return the page headers back to the "thesis" style

% Include the chapters of the thesis as separate files from the Chapters folder
% Uncomment the lines as you write the chapters

% Chapter 1

\chapter{Introduction} % Main chapter title

\label{Chapter1} % For referencing the chapter elsewhere, use \ref{Chapter1} 

%----------------------------------------------------------------------------------------

% Define some commands to keep the formatting separated from the content 
\newcommand{\keyword}[1]{\textbf{#1}}
\newcommand{\tabhead}[1]{\textbf{#1}}
\newcommand{\code}[1]{\texttt{#1}}
\newcommand{\file}[1]{\texttt{\bfseries#1}}
\newcommand{\option}[1]{\texttt{\itshape#1}}

%----------------------------------------------------------------------------------------

\section{Background and Context}

\subsection{Biomedical Text and Knowledge Sources}

The biomedical domain generates vast amounts of textual data through multiple channels including electronic health records (EHRs), scientific literature, and research databases \parencite{Kong2019}. This biomedical text comes in both \textbf{structured} and \textbf{unstructured} formats. Structured data refers to information stored in predefined fields (e.g. patient demographics, clinical codes, genomic sequences, chemical formulas) that can be readily queried. In contrast, \textbf{unstructured data} consists of free-form text such as clinical narratives, research abstracts, full-text articles, case reports, and experimental descriptions. Free-text biomedical documents are a natural and comprehensive way for healthcare providers and researchers to communicate complex medical knowledge, often capturing nuanced relationships between biomedical entities that do not fit neatly into structured fields \parencite{Laue2024}. For example, a research abstract may describe novel drug-disease interactions or a physician's note may document symptom severity in richer detail than any standardized code.

Notably, unstructured content constitutes the majority of biomedical documentation. By some estimates, \textbf{up to 80\% of healthcare and biomedical data is unstructured} text \parencite{Kong2019}. This includes clinical notes, research publications (PubMed contains over 35 million citations), experimental protocols, and case studies that do not reside in tabular databases but in narrative form. Such unstructured biomedical text is immensely valuable, as it provides comprehensive documentation of clinical reasoning, experimental findings, and scientific discoveries. However, because it lacks a predefined format, unstructured data is not directly usable by computers or standard query tools. Healthcare organizations and researchers are therefore faced with the challenge of leveraging these abundant textual documents to improve patient care, accelerate research, and enable evidence-based decision-making.

\subsection{Challenges with Unstructured Biomedical Text}

Working with unstructured biomedical text is inherently challenging. Free-text biomedical documents are \textbf{difficult for automated systems to interpret} due to their lack of consistent structure and the complexities of scientific and clinical language \parencite{Stenetorp2024}. Unlike structured database entries, biomedical text may contain context-dependent information, ambiguous terminology, and diverse writing styles across different domains (clinical, research, pharmaceutical). Indeed, biomedical text often includes \textbf{domain-specific abbreviations, technical jargon, complex chemical names, and variable nomenclature}, all of which increase the difficulty of parsing and analyzing the data. Important relationships might be embedded in complex sentence structures, making it hard to extract specific entity interactions without advanced text processing. As a result, analyzing unstructured biomedical literature has traditionally required labor-intensive manual review by domain experts or limited rule-based systems.

Another challenge is the inherent variability and complexity of biomedical language across different contexts. Research articles are not written with standardized vocabulary usage; different authors may describe the same biological process or clinical concept in varied ways. Moreover, biomedical text reflects the complexity of biological systems with their multifaceted interactions. In contrast to controlled vocabularies like medical ontologies, \textbf{biomedical literature presents irregular, multifaceted data}: biological entities often have multiple functions and interactions, and publications may discuss both established and novel findings. The technical style, specialized terminology, and the fact that findings are presented through different research perspectives make consistent information extraction problematic. These factors can confound algorithms attempting to identify entities (such as genes, diseases, or chemical compounds) and their relationships from raw text.

In addition, \textbf{scale and integration issues} pose further hurdles. The biomedical literature grows exponentially with thousands of new papers published daily \parencite{Kong2019}, leading to information overload; processing this volume is complex and time-consuming. The heterogeneity of data sources (research articles, clinical notes, patents, databases, etc.) adds complexity in combining unstructured with structured biomedical data. Data quality issues like inconsistent entity mentions or missing context can propagate errors in analysis. All these challenges underscore why much of the unstructured biomedical text remains underutilized for systematic knowledge discovery. However, given the wealth of knowledge contained in biomedical literature, there is strong incentive to overcome these difficulties through advanced computational methods. In recent years, the field of \textbf{biomedical natural language processing (NLP)} has gained wide interest as researchers apply machine learning and language models to parse and derive meaning from biomedical text. These developments set the stage for transforming unstructured biomedical documents into structured forms that can drive scientific insights and clinical applications.

%----------------------------------------------------------------------------------------

\section{Research Motivation}

\subsection{The Need for Structured Information Extraction}

Unlocking the value in unstructured biomedical text is a key motivation for this research. While researchers and clinicians rely on narrative documentation for communication and knowledge sharing, secondary use of this data (for systematic analysis, knowledge discovery, or automated reasoning) is limited by its unstructured nature. In order for biomedical information systems to fully utilize the vast literature and clinical documentation, the free-text portions must be converted into a structured, machine-readable form. \textbf{Information Extraction (IE)} is the crucial NLP task aimed at addressing this gap: it automatically identifies and encodes important pieces of information from unstructured text. By applying IE to biomedical literature, one can populate databases or knowledge bases with structured representations of scientific findings (e.g. identifying gene-disease associations mentioned in research abstracts and storing them as relationship triples in a knowledge graph).

There is a clear need for such \textbf{structured information extraction} to make use of the rich knowledge in biomedical texts. Studies have noted that much of the biomedical knowledge needed for drug discovery, clinical decision support, and research hypothesis generation is locked in free-text form \parencite{Liu2024}. Automating the extraction of entities (genes, diseases, chemicals, species) and their relationships from biomedical literature would enable this knowledge to be integrated with existing structured databases and ontologies. This, in turn, supports advanced applications: for example, drug discovery systems could identify novel therapeutic targets from literature that might otherwise be overlooked, and researchers could systematically aggregate evidence from thousands of clinical notes and publications to generate new hypotheses. Indeed, combining structured and unstructured biomedical data has been shown to improve the performance of predictive models and knowledge discovery systems, since literature often contains context and novel findings not yet captured in databases.

In summary, the motivation is that \textbf{structured representation of information yields more actionable and computable knowledge}. Rather than leaving free-text biomedical data underutilized, converting it into structured form (through IE and encoding) can facilitate systematic analysis and knowledge discovery. This thesis is driven by the recognition that developing robust methods to extract structured knowledge from unstructured biomedical text will significantly enhance our ability to leverage scientific literature for research acceleration and clinical insights.

\subsection{Graph Databases for Biomedical Knowledge}

Once information is extracted from text, an appropriate data representation is needed to organize and utilize it effectively. \textbf{Knowledge graphs} (implemented via graph databases) have emerged as a powerful paradigm for representing complex interconnected information in biomedicine \parencite{Milvus2025}. A knowledge graph consists of nodes (entities such as genes, diseases, chemical compounds, species, etc.) and edges (relationships between entities), forming a network of facts. This graph-based model aligns well with biomedical knowledge, which is inherently relational and heterogeneous. For instance, consider a disease that involves multiple genes, is treated by various drugs, and affects different biological pathways – a graph can naturally capture these many-to-many relationships (Gene→associated\_with→Disease; Drug→treats→Disease; Disease→affects→Pathway; and so on) in a way that traditional relational databases struggle to represent without complex join tables.

Another key motivation for using knowledge graphs is their support for \textbf{knowledge discovery and hypothesis generation} in biomedical research. By organizing biomedical knowledge as a graph of entities and relationships, researchers can build systems that not only retrieve information but also discover novel connections through graph traversal and reasoning. For example, a knowledge graph might help researchers identify potential drug repurposing opportunities by finding paths connecting known drugs to diseases through shared molecular targets or biological pathways. Previous research has demonstrated that integrating biomedical literature into knowledge graphs enhances complex reasoning and allows researchers to query scientific knowledge in more meaningful ways \parencite{Rotmensch2017}. Graph-based representations can unify knowledge from fragmented sources (different publications, databases, clinical records) and still preserve the connections needed for comprehensive biomedical understanding.

In summary, \textbf{graph databases provide the infrastructure to turn extracted biomedical facts into an "actionable" knowledge network}. By storing the output of information extraction in a knowledge graph, this research enables sophisticated queries, visualization of biomedical relationships, and improved knowledge discovery capabilities that leverage the full context of scientific literature and clinical data.

%----------------------------------------------------------------------------------------

\section{Research Question and Objectives}

\subsection{Primary Research Question}

The primary question driving this thesis is:

\textbf{How can unstructured biomedical text be transformed into an actionable knowledge graph using a multi-model approach for information extraction in healthcare and biomedical research?}

We seek to determine how biomedical documents (e.g., research abstracts, clinical notes, scientific articles) can be automatically analyzed to extract structured knowledge and represented in a useful knowledge graph. This encompasses developing a multi-model pipeline (specialized models for entity recognition, linking, and relationship extraction) and evaluating its effectiveness. The focus is achieving a transformation that is both \textbf{accurate} and \textbf{actionable} for real-world use cases like knowledge discovery, literature analysis, and biomedical research acceleration.

\subsection{Specific Objectives}

To address the primary research question, the following specific objectives are defined:

\begin{enumerate}
  \item \textbf{Design and implement a multi-model NLP pipeline for biomedical text.} This involves combining a named entity recognition (NER) model with entity linking and relationship extraction to process unstructured biomedical literature, identifying key biomedical entities (genes, diseases, chemicals, species) and their relationships from free text.

  \item \textbf{Leverage both domain-specific and general language models for information extraction.} We will compare a medical domain-specific pretrained model (\emph{MedGemma}) with a general-purpose model (\emph{Gemma}) for extracting relationships, evaluating whether domain-specific fine-tuning provides superior accuracy in biomedical relationship extraction.

  \item \textbf{Construct an actionable biomedical knowledge graph from extracted information.} Using identified entities and relations, build a knowledge graph in Neo4j by defining appropriate schema and automatically translating extraction outputs into Cypher queries that populate a structured, queryable representation of biomedical literature.

  \item \textbf{Ensure compliance with ethical standards and data usage policies.} Implement appropriate data handling practices throughout the pipeline, ensuring proper citation and usage of public biomedical datasets while maintaining research integrity and following established guidelines for biomedical text mining.

  \item \textbf{Evaluate the performance and utility of the proposed approach.} Develop evaluation frameworks measuring NER and relationship extraction accuracy using precision, recall, and F1 metrics against gold-standard biomedical datasets, plus knowledge graph quality assessment. Compare domain-specific versus general models through statistical analysis to quantify their impact on extraction accuracy and graph completeness.
\end{enumerate}

Through these objectives, the thesis systematically addresses the problem of converting unstructured biomedical text into a structured knowledge graph, from methodological development to evaluation of outcomes.

%----------------------------------------------------------------------------------------

\section{Ethical Considerations and Data Privacy}

\subsection{Responsible Use of Biomedical Data}

Biomedical text data encompasses both publicly available scientific literature and sensitive clinical information, requiring careful attention to ethical standards and privacy protection. While research literature is generally publicly accessible, clinical text contains sensitive personal health information that demands rigorous privacy safeguards. For any clinical data processing, a fundamental step is \textbf{de-identification} to remove or obscure all identifiers that could link back to individual patients, following guidelines like the HIPAA Privacy Rule's Safe Harbor standard.

In this research, we primarily utilize publicly available biomedical datasets such as \textbf{BioRED} \parencite{BioRED}, which contains scientific abstracts from PubMed that do not contain personal health information. These datasets are obtained through legitimate research channels and used according to their intended research applications. However, our methodology is designed to be applicable to clinical data as well, with appropriate privacy protections. Any clinical text would be assumed to be properly de-identified prior to processing, with remaining institutional identifiers handled through filtering or tokenization to ensure no real patient or provider identities are exposed.

\textbf{Data governance} principles are followed throughout, limiting data use strictly to stated research objectives of developing and evaluating biomedical information extraction methods. We maintain research integrity by properly citing all data sources and following established protocols for biomedical text mining. The focus remains on extracting general biomedical knowledge patterns rather than identifying specific individuals or sensitive information. Because the thesis is conducted in English, we ensure all datasets contain English-language content suitable for our methodological approach.

%----------------------------------------------------------------------------------------

\section{Thesis Contributions}

This thesis makes several contributions to the field of biomedical NLP and healthcare knowledge management:

\begin{itemize}
  \item \textbf{Integrated Multi-Model Extraction Pipeline:} We develop a novel pipeline integrating multiple NLP models to transform unstructured biomedical text into structured knowledge graphs. The pipeline combines biomedical named entity recognition, entity linking to external knowledge bases, and relation extraction using large language models, demonstrating orchestrated end-to-end biomedical information extraction.

  \item \textbf{Domain-Specific vs. General Language Model Comparison:} The research provides comparative analysis of a domain-specific model (\emph{MedGemma}) versus a general-purpose model (\emph{Gemma}) for extracting relationships from biomedical literature. We quantify their strengths and weaknesses on relationship extraction accuracy and knowledge graph quality, offering insights into domain specialization value for biomedical and healthcare applications.

  \item \textbf{Construction of an Actionable Biomedical Knowledge Graph:} The thesis presents design and implementation of a biomedical knowledge graph capturing scientific literature in machine-readable form. We contribute methodology for translating raw biomedical text into graph database entries with domain-tailored schema, evaluating the graph's ability to answer complex biomedical queries and demonstrating practical utility for knowledge discovery.

  \item \textbf{Empirical Evaluation and Open Insights:} We conduct comprehensive experiments evaluating each system component and the final knowledge graph using established biomedical benchmarks like BioRED \parencite{BioRED}. The thesis reports detailed results, error analyses, and case studies illuminating common challenges in biomedical information extraction \parencite{Hier2025}, contributing to broader understanding of effective approaches and remaining difficulties in transforming biomedical literature into structured knowledge.
\end{itemize}

Through the above contributions, the thesis advances both the methodology for biomedical information extraction and the practical considerations for deploying such techniques in real biomedical research and healthcare contexts. % Introduction
% Chapter 2

\chapter{Literature Review and Theoretical Background}
\label{chap:literature}

\section{Natural Language Processing in Healthcare}

\subsection{Evolution of Clinical NLP}

Natural language processing (NLP) techniques have been applied to clinical text for several decades. Early clinical NLP systems often relied on \textbf{rule-based} or \textbf{dictionary-driven} approaches. These systems used handcrafted rules and lookup lists of medical terms to extract information from clinical narratives. For example, an early system might flag any capitalized word followed by "disease" as a diagnosis, or use a dictionary of drug names to identify medications. Such rule-based methods, including those integrated into early clinical NLP tools (e.g., the Mayo Clinic's cTAKES or NLM's MetaMap), could achieve high precision on well-defined patterns. However, they were brittle – coverage was limited to what the rules enumerated, and maintaining a large set of rules for evolving clinical language proved labor-intensive.

By the late 1990s and 2000s, \textbf{statistical machine learning} approaches began to emerge in clinical NLP. Researchers started formulating problems like entity recognition or relation extraction as supervised learning tasks, moving away from purely manual rules. For instance, classification algorithms (Support Vector Machines, Maximum Entropy models, etc.) were trained on annotated clinical corpora to detect entities or relationships. Sequence labeling methods such as Hidden Markov Models (HMMs) and Conditional Random Fields (CRFs) became popular for tasks like \textbf{clinical named entity recognition (NER)}. These models use probabilities learned from data rather than fixed rules, allowing more flexibility. A CRF model could learn that "fever" or "pain" are likely problem entities if they occur in certain contexts, without an explicit dictionary. This statistical era greatly improved recall (catching more valid variations) but still required careful feature engineering by experts (e.g., designing features for medical abbreviations, context window words, part-of-speech tags, etc.).

Over the last decade, the field has undergone a \textbf{deep learning revolution}. Neural network–based models, which can automatically learn features from large amounts of text, have achieved state-of-the-art results in clinical NLP. Initial forays used \textbf{recurrent neural networks} (RNNs) or \textbf{convolutional neural networks} (CNNs) to process clinical text. For example, a bidirectional LSTM network with a CRF output layer can learn to recognize medical entities by capturing character-level and word-level patterns, surpassing the performance of CRF with manual features. The introduction of \textbf{word embeddings} (dense vector representations of words trained on unlabeled text) also boosted performance – models could leverage semantic similarities (e.g., that "myocardial" is similar to "cardiac") without explicit rules. By the late 2010s, the \textbf{transformer architecture} and in particular BERT-like models brought another leap. These models, pretrained on vast text corpora, allowed fine-tuning on relatively small clinical datasets with excellent results. The evolution of clinical NLP can thus be seen in phases: from rigid rule-based systems, to statistical ML with feature engineering, and now to end-to-end deep learning approaches. This progression has dramatically improved the ability to process free-text clinical notes, though challenges like data privacy, integration into workflows, and limited labeled data remain obstacles to broad deployment.

\subsection{Current State-of-the-Art Approaches}

State-of-the-art (SOTA) NLP in healthcare today is dominated by \textbf{advanced neural models}, especially transformer-based language models. In the realm of clinical text, researchers now routinely use \textbf{domain-specific pretrained models} to boost performance. For example, \textbf{BioBERT}, \textbf{ClinicalBERT}, \textbf{SciBERT}, and related models take the BERT architecture and continue pretraining it on biomedical literature or clinical note corpora. These models have learned the nuances of medical terminology from millions of words of domain text. As a result, they significantly outperform generic NLP models on tasks like medical NER and question answering. One study noted that BioBERT (trained on PubMed abstracts and PMC articles) \textbf{"significantly outperformed vanilla BERT on core biomedical tasks such as named entity recognition (NER), relation extraction, and question answering."} This has made biomedical transformers a de facto starting point for most new studies in clinical NLP.

In parallel, the NLP field at large has seen the rise of \textbf{large language models (LLMs)}, and these are beginning to influence clinical NLP. Models like GPT-3, GPT-4, PaLM, and others with tens or hundreds of billions of parameters have demonstrated impressive capabilities to understand and generate text. Researchers have started applying LLMs to clinical problems – for instance, prompting an LLM to extract medical information without task-specific training. Recent findings suggest that LLMs can achieve strong results in biomedical information extraction even in \textbf{zero-shot or few-shot settings}. An example is using GPT-4 to parse a radiology report and list key findings, or classify the relationship between a drug and an adverse effect in a sentence by just giving the model a suitable prompt. These models leverage the broad knowledge encoded during pretraining (some have seen portions of medical literature or general health information on the web). While not specialized to clinical text, their sheer scale often compensates, and fine-tuning or instructing them on biomedical tasks can yield high accuracy. It is now state-of-the-art to explore \emph{hybrid approaches}: using domain-specific models for certain tasks and general LLMs for others, or comparing their performance.

Despite the excitement around transformers and LLMs, practical state-of-the-art systems often combine multiple components and techniques. For example, a top-performing pipeline for clinical note understanding might use a \textbf{biomedical NER model} to find entities, a \textbf{rule-based or dictionary method} to handle specific patterns (e.g., lab values or abbreviations), and a \textbf{transformer-based relation extractor} to link entities. Integration with clinical knowledge bases (like UMLS) is also common in SOTA systems to ground extracted text in standardized medical concepts. In summary, current best approaches in clinical NLP leverage (1) pretrained biomedical language models for core NLP tasks, (2) large general models for generative or zero-shot capabilities, and (3) domain knowledge (ontologies, dictionaries, rules) to handle the idiosyncrasies of clinical data. This thesis builds upon these state-of-the-art trends, employing specialized models and knowledge resources to maximize extraction performance from clinical text.

\section{Named Entity Recognition in Medical Text}

\subsection{Traditional NER Methods}

\textbf{Named Entity Recognition (NER)} in medical text involves identifying spans of text that correspond to clinical concepts such as diseases, symptoms, medications, or procedures. Traditional approaches to medical NER were dominated by \textbf{rule-based and dictionary-based methods}. In a dictionary-based method, one compiles extensive lists of known entity names for each category (for example, a list of all drug names, a list of all human anatomy terms, etc.) and then scans the text for exact or fuzzy matches. Many early clinical NLP systems took this approach, often augmented with simple variants (e.g., allowing minor spelling differences or acronyms). Rule-based methods, on the other hand, rely on human-crafted linguistic patterns. For instance, a rule might be: "if the token 'diagnosed with' appears, the following words until a comma or verb are a Disease entity." Such rules leverage the observation that certain syntactic contexts reliably indicate an entity. \textbf{cTAKES}, an open-source clinical text processing tool, began with a heavy reliance on dictionary lookups (using the UMLS thesaurus) to identify entities, essentially a dictionary-based NER. Similarly, \textbf{MetaMap} (developed by the National Library of Medicine) uses a combination of dictionary lookup and lexical variation generation to map text to medical concepts, thereby performing entity recognition.

These traditional methods often achieved \textbf{high precision} in identifying entities that exactly match their lists or patterns – a drug dictionary will rarely misidentify a common word as a drug. However, they showed \textbf{limited recall and poor adaptability}. New or uncommon terminology, or mentions phrased in unexpected ways, would be missed. For example, a dictionary might contain "myocardial infarction" but not catch an abbreviated slang like "MI" unless explicitly added. Rule-based systems similarly struggle with the variability of clinical language; a slight change in wording can break a rule. Maintaining and updating these NER systems was difficult, as medical vocabulary continuously evolves (new drug names, emerging diseases, etc.) and language use varies across institutions and clinicians. Another challenge was \textbf{ambiguity} – many medical terms can play multiple roles. A rule-based system might mistakenly label "June" as a Person because months weren't anticipated, or label "mass" as a symptom when it might be a measurement unit, without more complex understanding.

Nonetheless, traditional NER methods laid the foundation and are still in use in constrained scenarios. They are \textbf{interpretable} (one can see why a term was recognized) and can be quickly deployed if one has a decent lexicon. In low-resource settings or specific subdomains (e.g., a hospital might maintain a custom list of its clinic names to pick out), these methods remain valuable. They also complement modern methods as a fallback or an ensemble component, because their precision can help validate or filter the more "eager" neural model outputs. Overall, however, the field has largely moved beyond purely rule-based NER due to the labor required and the need for higher recall and robustness.

\subsection{Deep Learning Approaches for Medical NER}

Deep learning approaches revolutionized medical NER by removing the need for manual feature engineering and by vastly improving the ability to generalize to new examples. Around the mid-2010s, researchers began applying \textbf{neural network models} to biomedical NER tasks. One common architecture was the \textbf{BiLSTM-CRF}: a bidirectional Long Short-Term Memory network feeding into a Conditional Random Field output layer. In this setup, word sequences are fed into the LSTM which captures context from both left and right (crucial for clinical text where context determines entity boundaries), and then a CRF layer jointly decodes the best sequence of labels (ensuring a consistent output like "B-problem, I-problem, O"). Such models quickly proved effective, outperforming earlier CRF models that relied on hand-crafted features. Studies found that incorporating \textbf{character-level CNN or LSTM} sub-networks (to capture morphological patterns, e.g., that many condition names end with "-itis") further improved recognition of rare or misspelled entities.

A key advantage of deep learning NER is the use of \textbf{word embeddings}. In medical NER, words like "aspirin" or "ibuprofen" will have vector representations that are close to each other if trained on clinical data, capturing the fact that both are medications. Early approaches used general embeddings (like word2vec or GloVe trained on Wikipedia) which helped, but soon \textbf{domain-specific embeddings} trained on medical corpora became available (e.g., embeddings from PubMed articles) and gave better results. These embeddings allow the model to leverage vast unlabeled text: even if "acetylsalicylic" (aspirin) was never in the NER training data, the model might still recognize it as a medication because its embedding is similar to known drugs.

By the late 2010s, \textbf{transformer-based models} took center stage. \textbf{BERT} (Bidirectional Encoder Representations from Transformers) introduced a powerful method of contextual representation, and researchers wasted no time applying it to biomedical NER. A common approach is to take a pretrained BERT (or a variant like BioBERT) and fine-tune it on a labeled medical NER dataset by adding a classifier for each token's label. This approach currently holds state-of-the-art performance on many NER benchmarks. For example, on the 2010 i2b2/VA clinical concept extraction challenge, fine-tuned transformers achieve substantially higher F1-scores than earlier RNN or CRF systems. Deep models are especially strong in handling \textbf{varied phrasing and contexts} – they can learn that "He was started on metformin" implies "metformin" is a Medication even if not explicitly capitalized or listed, and that "metformin" and "Glucophage" (brand name) refer to the same concept based on context usage.

Another trend in modern NER is using \textbf{multiple corpora and multi-task learning} to improve generalization. Recent systems like \textbf{HunFlair} combined training data from a wide range of biomedical NER datasets (covering genes, diseases, chemicals, etc.) to create a more robust model that doesn't overfit to one style of text \cite{Weber2023}. This addresses a common deep learning issue: a model trained only on, say, discharge summaries might falter on radiology reports; multi-corpus training alleviates this. There are also \emph{zero-shot} or \emph{few-shot} NER approaches emerging, where models like \textbf{GLiNER} (a zero-shot NER method) attempt to recognize entities without direct training on a specific dataset, by leveraging general language understanding or definitions of entities. While these are promising, they usually still trail well-trained supervised models in accuracy.

In summary, deep learning has become the dominant paradigm for medical NER due to its superior performance. The combination of powerful architectures (BiLSTM, CNN, Transformer), unsupervised pretraining (embeddings, language models), and larger annotated datasets has pushed concept extraction to new heights. Challenges remain – especially the need for large labeled datasets and ensuring the models handle edge cases correctly – but the gap between human annotator performance and automated NER has narrowed significantly in the clinical domain.

\subsection{Biomedical Language Models}

A major factor in the success of modern medical NLP is the creation of \textbf{biomedical language models}. These are large neural networks (typically transformer architectures) pretrained on huge amounts of biomedical text in an unsupervised manner, then fine-tuned for specific tasks. Notable examples include \textbf{BioBERT}, \textbf{ClinicalBERT}, \textbf{BlueBERT}, \textbf{SciBERT}, \textbf{PubMedBERT}, and more. The motivation is that while a general model like BERT learns English from Wikipedia and Books, it might not fully understand the specialized vocabulary and style of clinical narratives or biomedical literature. By further pretraining on domain text, the model picks up domain-specific terminology and context. For instance, BioBERT continued training BERT on 4.5 billion words from PubMed abstracts and PMC full-text articles. ClinicalBERT \cite{Alsentzer2019} trained on a corpus of EHR notes (MIMIC-III), exposing it to clinical shorthand and patient-related text. SciBERT was trained on scientific papers (biomedical and computer science) and even introduced a new vocabulary to better handle scientific terms.

These biomedical language models have unequivocally improved the performance of NLP tasks in healthcare. They serve as \textbf{foundational models} that can be fine-tuned with relatively little data. For example, a BioBERT-based NER model can outperform a vanilla BERT model on medical entity extraction by a large margin, because BioBERT "knows" words like "bradycardia" or "EKG" from its pretraining. In one comparative study, BioBERT was shown to \textbf{"largely outperform BERT and previous state-of-the-art models in a variety of biomedical text mining tasks."} This includes NER and relation extraction tasks on benchmarks such as the JNLPBA (bioentity recognition) or ChemProt (chemical-protein relation) dataset. Similarly, ClinicalBERT has been found to better capture context in clinical notes than general BERT, e.g., in extracting smoking status or identifying temporal events in a patient timeline. These models often come with domain-specific tokenizers – for instance, SciBERT's tokenizer includes more scientific terms – which further aids in representing domain phrases succinctly (treating "heart\_failure" as one token rather than two, for example).

Another development is \textbf{continual pretraining and task-specific pretraining}. For instance, a model like \textbf{BioMegatron} or \textbf{PubMedGPT} might be trained from scratch on biomedical text with even more parameters (bringing in the paradigm of GPT-style generative models to biomedical domain). There are models targeting clinical notes specifically (some being developed by companies or research groups that cannot be fully open due to patient data privacy). The general finding is that \textbf{domain knowledge in language models leads to better understanding}: a model that has read thousands of oncology papers will more likely flag "PD-1 inhibitor" as a Drug entity and know it's used in cancer treatment context, compared to a general model that might not recognize the term at all.

Biomedical language models do have limitations: they are usually large (requiring substantial computational resources), and they can still struggle with extremely rare or new terms (e.g., a newly approved drug that wasn't in the training data). They also inherit any biases or errors present in the biomedical literature. But overall, they represent the state-of-the-art starting point. In this thesis, we leverage such models for tasks like NER and potentially for relation extraction, expecting their domain-tuned knowledge to enhance performance. Furthermore, the comparison between a \textbf{medical-specific model} and a \textbf{general model} (the focus of our comparative analysis in Chapter 4) is essentially a comparison of the value of this domain pretraining – a topic of great interest in current research.

\section{Entity Linking Techniques}

\subsection{Knowledge Base Linking Methods}

Once named entities are recognized in text, the next step often needed is \textbf{entity linking} (also called entity normalization or grounding). This means mapping the text span (e.g., "aspirin") to a canonical identifier in a knowledge base (KB) (e.g., the concept code for Aspirin in a medical ontology). In the medical domain, entity linking is crucial for downstream applications because it resolves synonyms and variations to a standard reference (for example, linking "heart attack" and "myocardial infarction" to the same concept).

A variety of techniques have been developed for entity linking in biomedicine, ranging from simple string matching to complex machine learning models:

\begin{itemize}
\item \textbf{String Matching and Dictionary Lookup:} The most straightforward method is to use dictionaries of concept names from a knowledge base and attempt to match entity mentions to those names. \textbf{Traditional techniques for entity normalization performed recognition and linking in one step}, essentially scanning text for any substring that appears in the KB. Tools like \textbf{MetaMap} (for UMLS concepts) and \textbf{cTAKES dictionary lookup} fall in this category. They often allow some fuzziness (for example, ignoring case, minor typos, or word order) to match mentions to KB terms. Rules can be added to handle common variations (e.g., stripping plural 's' or matching acronyms). These methods are precision-oriented; they may miss entities that don't exactly match a known term. They also may struggle when a single mention has multiple possible mappings (ambiguity) without additional context.

\item \textbf{Rule-based and Heuristic Linking:} Some systems add heuristic rules to improve matching, essentially a step up from raw dictionary lookup. For instance, a rule might prefer a match to a concept whose semantic type fits the context (if "aspirin" is found in a Medication list section, prefer mapping it to a drug concept over a plant concept). Another heuristic approach is to use \textbf{co-occurrence}: if two candidate concepts often appear together in literature, that might influence selection. Traditional \textbf{rule-based normalization} systems like \textbf{GNormPlus} for gene names combined dictionary matching with custom post-processing rules to decide on the best gene ID. These approaches need domain expertise to create the rules and are often tailored to specific types of entities (genes, diseases, etc.).

\item \textbf{Machine Learning-based Linking:} Moving beyond hard rules, researchers introduced ML models for the linking step, especially as the number of candidate concepts can be very large (UMLS has millions of concepts). One approach is to formulate linking as a \textbf{ranking problem}: given a mention and a list of candidate concept IDs (those whose names or synonyms match closely), rank them to pick the best one. An early example is \textbf{DNorm} \cite{Leaman2013} which addressed disease name normalization by training a pairwise learning-to-rank model. It would take a mention and two candidate names at a time and learn which one is more likely correct. Features for such models include textual similarity scores, context vectors, concept popularity, etc. Another example, \textbf{TaggerOne} \cite{Leaman2016}, used a semi-Markov model to jointly recognize and normalize diseases and chemicals, learning parameters that consider both the context and the candidate concept's features. In these ML approaches, having training data (e.g., manually linked mentions in text) is critical. They often outperform pure dictionary lookups by learning to disambiguate based on usage context.

\item \textbf{Neural Embedding-based Linking:} With the advent of deep learning, a trend is to embed both mentions and knowledge base terms into a vector space and compute similarities. The idea is to learn an encoder (often a neural network) that converts any entity mention into a vector, and similarly each KB concept (perhaps by its name and definition) into a vector, such that the correct concept is closest to the mention in this space. Researchers have tried \textbf{character-level BiLSTMs} or CNNs to encode mentions and concept names. For example, Phan et al. \cite{Phan2019} train a char-BiLSTM that takes a mention string and a candidate concept name and learns to output high similarity if they match. They use multiple objectives, like one that maximizes mention-concept similarity and another that accounts for context coherence. Similarly, other work used CNNs on character n-grams of mentions with a loss function to differentiate the correct concept from incorrect ones.

\item \textbf{Pretrained Language Model for Linking:} The latest methods use transformers (BERT-based models) for entity linking. One prominent example is \textbf{BioSyn} \cite{Sung2020}, which uses BioBERT as an encoder for mentions and concept names. During training, BioSyn fine-tunes BioBERT such that the vector for a mention is close to vectors of its true concept's names and far from others. At inference, BioSyn can encode a mention and then do a fast nearest-neighbor search among all concept embeddings to find the best match. These methods achieve very high accuracy on biomedical linking benchmarks because the language model captures a rich understanding of synonyms and context. Some extensions incorporate hierarchical knowledge (e.g., if two concepts are parent-child in ontology, use that as a feature or constraint).
\end{itemize}

Each of these methods has trade-offs. Simpler methods (string match) are fast and don't require training data, but can't resolve ambiguity well. Advanced methods (neural linking) are more accurate in principle but require significant computation and annotated training data. Moreover, some neural methods struggle with the \textbf{scale} of biomedical KBs – performing a vector search over millions of concepts can be slow or memory-intensive, which is why approximate search or pruning strategies are used. In practice, many systems adopt a \textbf{hybrid approach}: e.g., use quick string matching to get candidates, then a learned model to pick the best, combining speed with accuracy. Our work specifically leverages \textbf{SciSpacy's} linking (described below) for candidate generation, which is a strong unsupervised method, and we incorporate confidence scoring and domain rules to refine the results.

\subsection{UMLS and MeSH Knowledge Bases}

In the biomedical domain, the two predominant knowledge bases used for entity linking in text are \textbf{UMLS} and \textbf{MeSH}, among others. It is important to understand their nature and why they are favored over general-purpose KBs (like Wikipedia or Wikidata) for clinical NLP.

The \textbf{Unified Medical Language System (UMLS)} is a comprehensive metathesaurus maintained by the U.S. National Library of Medicine. It aggregates over 150 biomedical vocabularies and ontologies into a unified framework. UMLS provides a concept unique identifier (CUI) for each concept, and links that concept to all its names (synonyms, abbreviations, variants in different source vocabularies) and relevant semantic types. For example, the drug aspirin has a CUI (C0004057) which ties together names like "Aspirin", "Acetylsalicylic Acid", brand names, etc., and identifies it as a pharmacologic substance (semantic type). UMLS is \textbf{massive}, containing over \textbf{3 million concepts} drawn from sources like SNOMED CT, RxNorm, ICD, MeSH, etc.. This breadth is one of its strengths – it covers clinical terminology (diseases, procedures), medications, devices, organisms, and even health administration terms. When linking entities, mapping to UMLS CUIs allows integration with electronic health records and clinical decision support systems, which often use these standard codes. It also facilitates aggregation: one can link "heart attack" in text to a UMLS CUI and then know that it corresponds to ICD-10 code I21 (for myocardial infarction), etc.

\textbf{MeSH (Medical Subject Headings)} is another widely used vocabulary, primarily for indexing journal articles (PubMed). MeSH is a curated hierarchical taxonomy of biomedical terms, with about \textbf{30,000 main headings}. These include diseases, chemicals, anatomical terms, etc., organized into a tree (so "Heart Diseases" might be a category with children like "Myocardial Infarction", etc.). MeSH terms are used by librarians to tag articles for easier retrieval. In NLP, MeSH can be used as a linking target, especially for biomedical literature text (e.g., linking an article's key terms to MeSH can help find related articles). SciSpacy, for instance, supports linking to MeSH as an alternative to UMLS. MeSH is \textbf{smaller and more curated}; each concept is well-defined and has synonyms listed, but it won't include every clinical term (for example, very specific procedure names or abbreviations used in clinical notes may not be MeSH terms).

Using medical-specific KBs like UMLS/MeSH has clear advantages over general-purpose knowledge bases for this task. First, \textbf{coverage}: general KBs (like Wikipedia) might not list every laboratory test name or obscure medical acronym, whereas UMLS likely does (through one of its source vocabularies). Second, \textbf{terminology consistency}: UMLS provides a unified identifier for concepts that appear under different names in different contexts. This is crucial in clinical data where synonyms and acronyms abound. Third, \textbf{domain relevance}: UMLS/MeSH have semantic types and relationships tailored to medicine (e.g., "Drug A treats Disease B" relationships exist in UMLS through RxNorm or SNOMED CT relations), enabling richer graph construction later. In contrast, a general KB might have the drug and disease as unrelated entries or not capture the treatment relation explicitly.

However, UMLS is \emph{so} comprehensive that it introduces ambiguity; many concepts and names overlap. This makes linking challenging – hence the need for algorithms as described. Additionally, UMLS requires a license (free for research but with a process), which is another reason why specialized tools exist to handle it (they often come with the UMLS data pre-packaged for convenience). In our project, we predominantly use \textbf{UMLS} as the linking target, given its broad coverage, and occasionally reference \textbf{MeSH} when discussing alternate approaches or quality of concepts.

\subsection{SciSpacy's TF-IDF Character N-gram Matching}

\textbf{SciSpacy} \cite{Neumann2019} is an open-source library developed by the Allen Institute for AI that extends the popular spaCy NLP framework to scientific and biomedical text. It includes pre-trained models for biomedical NER and, importantly, an \textbf{EntityLinker} component that can link text spans to knowledge base entries. Our system leverages SciSpacy for entity linking, so we detail its approach here.

SciSpacy's EntityLinker uses a \textbf{sparse vector model based on TF-IDF over character n-grams} to perform linking. The process can be summarized as follows: SciSpacy comes with a pre-built knowledge base index (for example, an index of UMLS concept names). Each concept name (and its synonyms/aliases) in the KB is indexed by the character 3-grams it contains. For instance, the term "aspirin" would be broken into character trigrams: "asp", "spi", "pir", "iri", "rin". This is done for all terms, yielding a high-dimensional space of n-gram features. When an entity mention is extracted from text (say, "Aspirin 325mg"), SciSpacy takes the mention text ("Aspirin") and computes its TF-IDF weighted character n-gram vector. It then performs an approximate nearest neighbors search in the index of KB terms to find the closest matches by cosine similarity. In essence, it's looking for the concept whose name shares the most overlapping n-grams with the mention, with weighting that downranks very common n-grams. This method is a robust string matching strategy: it can handle minor spelling differences (because many 3-letter sequences will still match) and partial matches. It's similar to other string matching tools like \textbf{QuickUMLS}, which uses character n-gram overlap and heuristics.

The SciSpacy linker returns, for each entity span, a list of candidate KB identifiers along with a \textbf{similarity score} (confidence). By default, it may return up to a certain number (e.g., top 5) candidates for each mention, sorted by the TF-IDF similarity score. Each candidate has the KB concept ID (like a UMLS CUI) and the similarity score. For example, if the mention is "COPD", SciSpacy might return: CUI for "Chronic Obstructive Pulmonary Disease" (score 0.95), CUI for "Compulsive Obsessive Personality Disorder" (score 0.60) – illustrating a correct and a less likely candidate. It's then up to the user/system to decide how to use these; one might take only the top hit if the score is above a threshold, etc. SciSpacy provides the knowledge base object so one can also retrieve the concept name, definition, etc., for each CUI.

One valuable feature of SciSpacy is its integration of \textbf{abbreviation detection}. Clinical text is rife with abbreviations (e.g., "HTN" for hypertension, "T2DM" for type 2 diabetes). SciSpacy includes an \textbf{AbbreviationDetector} component based on the Schwartz \& Hearst algorithm. This component scans the document for patterns like "Full Form (ABBR)" and learns that the abbreviation "ABBR" stands for "Full Form" in that document. It also can infer abbreviations without explicit definitions by statistical means (though defined ones are easier). When the EntityLinker is configured with \texttt{resolve\_abbreviations=True}, it will attempt to link the \emph{long form} of an abbreviation rather than the short form. For example, if a note says "Patient has Chronic Obstructive Pulmonary Disease (COPD). COPD is worsening." – the AbbreviationDetector will note "COPD" = "Chronic Obstructive Pulmonary Disease". Then when linking, SciSpacy will treat the second "COPD" mention as if it were "Chronic Obstructive Pulmonary Disease", ensuring it links to the correct concept rather than trying to match "COPD" literally (which it could also do, but the long form provides more clarity). This greatly improves linking accuracy for abbreviations and acronyms.

The simplicity of SciSpacy's TF-IDF n-gram approach is actually a strength in practice: it's \emph{fast}, and it doesn't require any training data or machine learning to use. It can be applied to very large KBs (the heavy lifting is done by efficient nearest-neighbor search libraries). However, it does have limitations – it purely matches on surface forms and does not use context beyond the mention string itself. For example, "cold" could link to a disease (common cold) or a symptom (feeling cold) or a medication (Cold tablets), and SciSpacy will just return whichever term name is most similar, perhaps many candidates if "cold" appears in many concept names, without knowing which is meant in the text. That's where we might incorporate \textbf{confidence thresholds or post-filters}. SciSpacy allows filtering out candidates that lack a definition or are very low score, etc., which can be tuned. We leverage these features in our pipeline: we use SciSpacy to get candidate UMLS CUIs for each entity, then apply a confidence score cutoff (and some custom logic like preferring certain semantic types) to finalize the linking. Overall, SciSpacy's TF-IDF n-gram method provides a solid baseline for entity linking, capturing a large majority of cases especially when the entity mention closely resembles the canonical name or a known synonym. Harder cases (abbreviations, ambiguous short terms) are where additional strategies are needed, but SciSpacy gives us a strong starting point without needing to train a complex model from scratch.

\subsection{Challenges in Medical Entity Linking}

Linking clinical entities to a knowledge base is a challenging task due to several inherent issues in medical text and the nature of biomedical knowledge bases:

\begin{itemize}
\item \textbf{Synonymy and Term Variation:} Medical concepts often have numerous ways to be expressed. A single disorder might have a long formal name, several shorthand names, and colloquial descriptions. For instance, "heart attack", "myocardial infarction", "AMI", and "cardiac infarct" all refer to the same concept. A linking system must recognize these variants and unify them. UMLS helps by listing synonyms, but the text might contain something not in the synonym list (e.g., a misspelling or a very specific description). Ensuring that different surface forms map to one concept (and not accidentally to different but related concepts) is tricky. This is essentially a \textbf{recall} challenge – failing to link an entity because the exact wording wasn't seen before.

\item \textbf{Ambiguity (Polysemy):} Many medical abbreviations and terms are ambiguous. "RA" could mean rheumatoid arthritis, right atrium, renal artery, etc., depending on context. Even non-abbreviations can be ambiguous ("discharge" could be a verb or a noun (fluid), "mass" could be a tumor or an amount). When a mention can map to multiple KB entries, the linker has to pick the correct one. This is a \textbf{precision} challenge – linking to the wrong concept is often worse than not linking at all. Disambiguation requires using the context in which the term appears (neighboring words, document metadata, etc.), which simple string-based methods do not fully exploit. For example, the mention "COPD" in a respiratory clinic note clearly should link to lung disease, whereas in a psychology note "COPD" is unlikely to appear at all. Context-aware linking (as some neural methods attempt) is an ongoing challenge, especially with limited training data for those methods.

\item \textbf{Data Scarcity for Training:} Unlike NER where one can sometimes obtain labeled data by annotation, linking requires \textbf{mapping to concepts}, which is a more complex annotation task. Creating a gold-standard corpus where every entity mention is mapped to a concept like a CUI is time-consuming and requires domain experts. There have been some shared tasks (e.g., the 2019 n2c2 challenge had a concept normalization track for disorders) producing datasets, but these are relatively small and specific. The lack of large, general training corpora for entity linking means supervised machine learning approaches can overfit or fail to generalize. This is one reason why dictionary and heuristic approaches are still popular – they don't need training data. Research into distant supervision (using dictionary matches as pseudo-training data) or leveraging unlabeled data is ongoing to mitigate this.

\item \textbf{Knowledge Base Coverage and Granularity:} UMLS is huge, but not perfect. There might be concepts in the text that are \textbf{not present} in the KB or not at the right granularity. For example, UMLS might have a concept for "Type 2 diabetes" and "Type 1 diabetes", but what if a note says "brittle diabetes" (an old term for unstable Type 1 diabetes)? There may or may not be a direct concept for that. Choosing a concept that is too general or too specific is a challenge – should "brittle diabetes" map to Type 1 diabetes concept, or just diabetes mellitus unspecified? Such decisions can require nuanced understanding. Moreover, UMLS merges many sources, which can lead to duplicative or very similar concepts; linking systems might get confused between two codes that both essentially mean the same thing. Deciding which one is "correct" may be arbitrary (this is where application context matters – if you plan to later use the links for a particular purpose, you might prefer one vocabulary's concept over another).

\item \textbf{Abbreviations and Shorthand:} As noted, abbreviations are extremely common. While SciSpacy and similar can detect many, it's not guaranteed. Some abbreviations are ambiguous (context needed), and some are rare or user-specific. For example, a doctor might write "pt c/o CP" which means "patient complains of chest pain." Linking "CP" to "Chest Pain" requires not just knowing the expansion but also that chest pain is a symptom concept in the KB. Handling abbreviations often needs a multi-step approach: detect and expand the abbreviation, then link the expansion. If abbreviation expansion fails, linking the short form directly is low-confidence because it might map to dozens of concepts (CP could map to Cerebral Palsy, for instance, aside from Chest Pain). This remains a pain point in clinical entity linking.

\item \textbf{Context and Relation Constraints:} Often, whether a mention should be linked or what it links to depends on context or implicit relations. E.g., in "family history of breast cancer", the mention "breast cancer" should perhaps link to the concept of breast cancer \emph{but flagged as family history} (some systems treat this differently). Or in "denies chest pain", the concept is chest pain but the context is negated. While linking typically ignores negation (linking "chest pain" to the concept regardless), certain applications might want to encode that the patient \emph{does not have} that problem. Some recent research tries joint entity linking and relation extraction (like identifying negated concepts), but generally linking algorithms by themselves don't do this. It's a challenge left to either pre-processing (not marking negated concepts at all) or post-processing the graph.

\item \textbf{Scalability and Performance:} As mentioned, a high-end method like using BERT to encode every mention and compare with every concept would be extremely slow with millions of concepts. This scalability issue means that sometimes a simpler method is chosen as a compromise. Even with approximate nearest neighbor methods, memory usage is high for storing embeddings of all concepts. So there's a practical challenge of making entity linking both \emph{fast and accurate}. This is an active engineering problem – for our pipeline, we needed to ensure that linking each note does not become a bottleneck, which is one reason we opted for SciSpacy's efficient approach and then deal with a smaller set of candidates.
\end{itemize}

Due to these challenges, \textbf{no single linking method is perfect}, and mistakes can propagate to later stages (for example, linking "Cold" to the concept of "Common Cold" when the text meant "cold sensation" could lead a knowledge graph to erroneously indicate the patient had a cold infection). Our thesis addresses some of these by using confidence scoring (not accepting low-score links), abbreviation handling, and by constraining the types of entities we link (focusing on ones most useful for the graph). It's also worth noting that because of the above difficulties, many state-of-the-art tools still include a \textbf{manual or rule-based component} for linking certain entities or defaulting to dictionary matches when the fancy model is unsure. This acknowledgement in the community underscores that a combined approach (where simple methods handle easy cases and complex methods handle hard cases) is often the best practical solution – a philosophy we carry into our multi-model system.

\section{From Relationship Extraction to Knowledge Graphs}

\subsection{Relationship Extraction Methods}

Identifying relationships between medical entities is the step that transforms isolated pieces of information into structured knowledge. In text, a \textbf{relationship extraction (RE)} system might detect, for example, that a certain drug is indicated for a disease, or that one clinical finding is a symptom of a condition mentioned elsewhere in the text. Extracting such relations from unstructured text allows us to then build edges in a knowledge graph (connecting the nodes that were identified via NER and linking).

\subsubsection{Rule-Based Approaches}

Early approaches to relation extraction in clinical text were predominantly \textbf{rule-based or pattern-based}. These systems rely on human experts to define patterns that indicate a relationship. For instance, a simple rule might be: \emph{if a medication name and a condition appear in the same sentence with the word "for" or "to treat" in between, infer a Treats relation}. An example rule could target a template like "<Drug> for <Condition>" (e.g., "Prescribed \textbf{metformin} for \textbf{diabetes}"). Other patterns might use punctuation or specific trigger verbs ("caused", "due to", "associated with") to link an entity pair. Syntactic patterns can also be used: using a dependency parse of the sentence, one might write a rule that if an entity of type Disease is the direct object of "diagnosed with", link it as a Diagnosis relation for the patient.

Rule-based RE has the advantage of being \textbf{precision-oriented and interpretable}. If a relation is extracted, one can trace exactly which rule fired. In critical applications (like identifying adverse events), a carefully crafted rule might be preferred for its reliability. Furthermore, rules can incorporate domain knowledge easily; for example, you might add a rule that "if X is a Finding and Y is a Disease and they are connected by 'of', then X is a symptom of Y". This leverages knowledge of grammatical constructs common in clinical writing.

However, as with rule-based NER, the \textbf{coverage is a major issue}. Language is very variable, and it's impossible to enumerate all ways a relationship might be expressed. Co-occurrence in the same sentence might signal a relation, but not always (it could be coincidental). Conversely, a relation might be implied across sentences (which simple intra-sentence rules would miss). Pattern-based methods also struggle with \textbf{negation and temporal expressions} that modulate relationships. For instance, the sentence "No evidence of metastasis to the lung" contains "metastasis" and "lung" which might match a pattern "X metastasis to Y" normally indicating a spreads-to relation, but here it's negated. Handling such nuances requires either very complex rules or a hybrid with some semantic understanding. Indeed, a known drawback is \textbf{low recall}: pattern-based methods often only capture a subset of true relations, as noted by recent studies \cite{Laue2024}. They tend to miss relations phrased in unexpected ways and can be brittle if the text deviates even slightly from the anticipated structure.

Despite that, rule-based approaches are still used in certain settings. For example, a hospital might implement a rule-based system to flag specific events (like "allergy to X" relations from notes) because it's easier to maintain and verify. Also, rule-based extraction can complement statistical methods – some systems do a first pass with rules to catch obvious cases with high precision, and then handle the rest with machine learning. In summary, rule-based RE methods laid important groundwork and are characterized by high precision and transparency, but require extensive knowledge engineering and don't scale well to the diversity of real-world text.

\subsubsection{Machine Learning Techniques}

As annotated datasets of clinical text became available (e.g., through shared tasks like i2b2 challenges), \textbf{machine learning (ML)} approaches to relation extraction gained prominence. In these approaches, rather than manually specifying what linguistic patterns indicate a relation, one trains a model on examples of text where entities are labeled and the relation between them (or lack thereof) is annotated. The model then learns to predict relations in new text.

\textbf{Feature-based ML:} Early ML methods for clinical RE often used classifiers such as Support Vector Machines (SVM) or Logistic Regression. These models require the text to be converted into a feature vector. Researchers engineered features capturing various signals: the words between the two entities (for example, presence of words like "for", "due to" might signal certain relations), the order of entities (does treatment precede disease or vice versa), entity types (Drug-Disease pair vs. Disease-Disease pair), syntactic parse features (like the shortest path in the dependency tree between the two entities), and section information (mention in the "Medication" section of a note vs. "Family History" section changes interpretation). A concrete example: to determine if a Medication and a Problem are in an "adverse reaction" relation, features might include: whether the problem is immediately after the med ("Penicillin allergy"), whether the verb "causes" or "induced" connects them, etc. Using such features, an SVM could be trained to classify yes/no for a specific relation type. Some systems trained one classifier per relation type, others a single multi-class classifier to decide which relation (including "no relation"). Feature-based ML was quite successful in competitions; many top entries in the 2010 and 2012 i2b2 challenge for relation extraction were SVMs with cleverly designed features. They substantially beat pattern-matching in recall, while maintaining decent precision. For instance, an SVM could learn a generalization like "if entity1 is a drug and entity2 is a disease and 'for' is in between, it's likely a treatment relation" without someone explicitly coding that rule, by observing many training examples.

One limitation, however, is that these models \textbf{don't generalize well to contexts not represented in features}. If an important cue isn't captured as a feature, the model can't learn from it. Plus, they tended to require a lot of domain-specific tweaking – essentially moving the burden from writing rules to writing good features.

\textbf{Neural Network ML:} As with NER, relation extraction also moved towards neural approaches around mid-2010s. Instead of manual feature extraction, models like \textbf{Convolutional Neural Networks (CNN)} or \textbf{Recurrent Neural Networks (RNNs)} were applied to the sequence of words between and around the target entities. A classic architecture is: represent the sentence (with marked entity positions) as an embedding sequence, feed into a CNN or LSTM, and then have the network output a relation class. The network's internal weights will learn to emphasize certain patterns. For example, a CNN with filters might learn that "X \emph{was started on} Y" is a strong indicator of a medication X being used for condition Y, without us explicitly coding that phrase. RNNs (like LSTMs) can capture longer dependencies and have been used to detect relations spanning clauses or sentences. More recently, transformer models (like BERT again) have been fine-tuned for relation classification: one common method is to insert special tokens around the two entities in the text and then feed it to BERT and use the output to classify the relation. This has achieved state-of-the-art results on many benchmarks. The fine-tuned BERT implicitly leverages both local context and global semantics to decide if, say, a "Treatment" relation exists between a drug and a condition in a sentence.

The ML approaches (especially neural ones) greatly improved \textbf{recall and adaptability}. They can find relations expressed in subtle ways that rules might miss, and they don't require human experts to enumerate patterns. For example, a neural model might catch that in "She responded well to metformin, her diabetes is now controlled," there is an implicit treatment relationship, even though it's not in a simple pattern. However, ML models, particularly deep ones, need substantial labeled data to train effectively. In clinical domains, getting thousands of examples of each relation type annotated is challenging. As a result, many neural models are trained on combined datasets or even on \emph{synthetically generated} data to supplement. There's also the issue of \textbf{error interpretability}: when a learned model makes a mistake (e.g., linking a symptom to the wrong condition), it can be hard to understand why, making debugging and refining difficult.

That said, the current trend is clearly towards neural RE when data is available, as it consistently outperforms feature-based methods. Some hybrid systems feed outputs of neural models into rule-based frameworks or vice versa to get the best of both worlds. For instance, one might use a quick classifier to filter obvious non-relations and then apply precise rules on the remainder, or use rule outputs as additional features for an ML model. In our project, for the \textbf{relationship extraction component}, we explore using large language models (see next section) – essentially taking the neural approach to the extreme by leveraging models trained on enormous general text corpora.

\subsubsection{Large Language Models for Relationships}

The latest frontier in relationship extraction is the use of \textbf{Large Language Models (LLMs)}, such as GPT-3, GPT-4, PaLM, Bloom, etc. These models are trained on diverse internet-scale text and exhibit an ability to perform tasks via prompting – even tasks they weren't explicitly trained for – by virtue of their general language understanding and world knowledge. In the context of biomedical relation extraction, researchers have started investigating how well LLMs can identify and extract relationships without explicit task-specific training. The findings so far are promising: \textbf{"Large Language Models have demonstrated impressive performance in biomedical relation extraction, even in zero-shot scenarios."} \cite{Laskar2025}.

There are a couple of ways LLMs are used for RE:

\begin{itemize}
\item \textbf{Prompt-based Classification:} You can present the LLM with a prompt that describes the relation extraction task and the specific instance. For example, to classify the relation between a gene and a disease in a sentence, one might prompt GPT-4 with: \emph{"Task: Determine if there is a positive, negative, or no relation between the gene [BRCA1] and the disease [breast cancer] in the following sentence: 'The BRCA1 mutation is associated with a higher risk of breast cancer.' Answer with '1' for positive relation and '0' for no relation."}. The model then outputs the label. This approach effectively turns the RE task into a natural question-answer format. Researchers have found that with carefully engineered prompts (and possibly a few examples provided in the prompt for few-shot learning), LLMs can achieve accuracy comparable to traditional supervised models on some benchmarks. The advantage is that this requires \textbf{no training phase} on task-specific data – the model is leveraged as-is. This is extremely useful in clinical RE because, as noted, large annotated datasets are scarce. An LLM like GPT might already "know" typical relations (e.g., that certain drugs treat certain diseases) and pick up on textual cues to apply that knowledge.

\item \textbf{Prompt-based Extraction (Generative):} Instead of classifying from given options, we can ask the LLM to directly extract relations in a more open-ended way. For instance: \emph{"Extract all <drug, condition> treatment pairs from the text: …"}. The model could generate a list of pairs it believes are in a treatment relationship. Another example: given a clinical note, ask the model to output structured data (perhaps as JSON) listing each problem and any associated tests or medications mentioned. This uses the generative nature of LLMs to do information extraction in a flexible way. Tools like GPT-4, when properly prompted, can sometimes parse complex text and output triples or relationships with surprising accuracy. They inherently do a form of joint understanding of NER and RE – because they identify entities and the relation simultaneously when generating the output.
\end{itemize}

Using LLMs for RE also brings some \textbf{new challenges}. Firstly, evaluation is tricky: LLMs might express the relationship in words different from the gold standard, which makes automatic evaluation hard (hence research on using LLMs themselves as evaluators). Secondly, they can \textbf{hallucinate} – produce a relation that sounds plausible but isn't actually stated in the text. For example, given a patient case, an LLM might "fill in" a likely treatment that the text didn't explicitly mention. This is dangerous in a factual extraction context. Ensuring the model sticks to the text evidence is an area of active research (techniques like chain-of-thought prompting or adding instructions to not include unmentioned facts). Thirdly, \textbf{privacy and deployment}: Many powerful LLMs are accessible only via cloud APIs, which is problematic for patient data. This has spurred interest in smaller domain-specific LLMs that could be run locally on hospital data (the "MedGPT" sort of efforts).

Despite these issues, the ability of LLMs to perform zero-shot or few-shot RE is a potential game-changer. It can dramatically reduce development time (no need to train a model for each new relation type; just prompt it appropriately) and can adapt to multiple tasks easily. Our thesis specifically compares a \textbf{medical-specific model} (MedGemma3-4b) with a \textbf{general-purpose model} (Gemma3-4b) for relation extraction, which aligns with examining this trend: How much does a general LLM know about clinical relationships out-of-the-box, versus a model tuned with medical knowledge? Prior work indicates general models can be strong but domain-tuned models still have an edge in specialized accuracy. For instance, GPT-3 might do well, but a smaller model fine-tuned on biomedical text might catch nuances GPT-3 misses. We will explore this in Chapter 4. In summary, LLMs represent the cutting-edge approach to extracting relationships, offering flexibility and leveraging massive implicit knowledge, and are likely to play an increasing role in clinical information extraction systems.

\subsection{Knowledge Graph Construction}

\subsubsection{Graph Database Technologies}

A \textbf{knowledge graph} is essentially a network of entities (nodes) linked by relationships (edges) that represent facts or assertions. Once we extract entities and relationships from text, we need to store them in a structured way for querying and analysis. This is where \textbf{graph database technologies} come into play. Unlike traditional relational databases that use tables, graph databases are designed to store and query data in terms of nodes and edges, which is a natural fit for knowledge graphs.

There are two main paradigms for graph data management: \textbf{RDF triple stores} and \textbf{property graph databases}.

\begin{itemize}
\item \textbf{RDF Triple Stores:} These databases store data as triples (subject, predicate, object), in line with the Resource Description Framework (RDF) standard from the semantic web community. For example, (Aspirin, treats, Diabetes) would be a triple. Triple stores (like Apache Jena TDB, Virtuoso, GraphDB) often use the SPARQL query language to retrieve data. SPARQL queries look somewhat like SQL but are tailored to graph patterns – you specify a triple pattern with variables and the engine finds matches in the data. RDF systems excel at enforcing a schema/ontology (you can have formal ontologies in OWL that the data adheres to) and at integrating data from multiple sources via shared identifiers. They're popular in academic knowledge graph work and where semantic interoperability is key (like linking to external ontologies).

\item \textbf{Property Graph Databases:} These are more commonly used in industry and practical applications. In a property graph model, each \textbf{node} can have labels and properties (key-value pairs), and each \textbf{relationship (edge)} has a type and can also have properties. For example, we might have a node of type "Medication" with properties like \texttt{name="Aspirin"} and \texttt{dosage="325mg"} (if storing patient-specific data), and a node of type "Condition" with \texttt{name="Diabetes Mellitus"} and perhaps a severity property. An edge of type "TREATS" from the Aspirin node to the Diabetes node could have a property \texttt{source="doctor\_notes"} indicating how we know this (from text). This property graph model is very flexible – we can attach various contextual data directly on nodes/edges. \textbf{Neo4j} is a leading database in this category and is known for being user-friendly and highly optimized for graph queries. Other examples include Amazon Neptune (which actually supports both RDF and property modes), JanusGraph, and TigerGraph.
\end{itemize}

Graph databases allow efficient traversal of the graph. For instance, a query like "find all medications that treat diseases which have symptom X" would involve traversing from a symptom node X, to disease nodes connected by "hasSymptom" edges, then from those diseases out via "treatedBy" or "treated\_with" edges to medication nodes. Doing this with SQL would be complex and slow if the data is heavily relational and many joins are needed, whereas graph DBs are optimized for such multi-hop queries.

Moreover, graph databases align with how we conceptualize knowledge: they can directly store an edge saying \textbf{Drug->treats->Disease}, which is more intuitive than having to indirectly connect them via foreign keys in tables. In a knowledge graph, relationships are first-class citizens – you can query them, filter on them (e.g., find all "inhibits" relations between genes and proteins, etc.), and even put constraints on paths.

For our project, the property graph model is suitable because we are building a graph of \emph{specific extracted facts} (not just a generic ontology). Each extracted entity becomes a node, with properties like its type (Problem, Test, Treatment, etc.) and maybe the UMLS CUI as an attribute. Each extracted relation becomes an edge, with a type (like "ASSOCIATED\_WITH" for symptom-disease, "TREATS" for drug-disease, etc.) and possibly a confidence score property or provenance (which note it came from). This structured representation then lives in a graph database so we can query it to answer questions or analyze the network.

In summary, graph database technologies provide the backbone for \textbf{storing} and \textbf{querying} the knowledge graph that results from our NLP pipeline. They handle the heavy lifting of managing connections at scale. The choice between RDF vs property graph often comes down to use case; we have chosen a property graph approach (Neo4j) for its straightforwardness and the ability to easily attach attributes to nodes/edges which is useful for tracking metadata of extracted info.

\subsubsection{Neo4j and Cypher Query Language}

\textbf{Neo4j} is a widely-used graph database that implements the property graph model and has become a standard for many knowledge graph projects. It's known for its performance, developer-friendly interface, and a rich ecosystem of tools. In the context of our thesis, Neo4j serves as the platform where the extracted knowledge graph is built and stored. Here's why we chose Neo4j and how we use it:

\begin{itemize}
\item \textbf{Data Model:} In Neo4j, data is stored as nodes and relationships. We design a schema (or \emph{graph model}) where, for example, we might have labels like \texttt{:Person}, \texttt{:Condition}, \texttt{:Drug}, etc., for nodes, and relationship types like \texttt{DIAGNOSED\_WITH}, \texttt{TREATS}, \texttt{CAUSES}, etc., for edges. Each node can have one or more labels and a set of properties; each relationship has a type and can also carry properties. This is very convenient to represent complex clinical data – e.g., a patient node can be linked to multiple condition nodes via \texttt{HAS\_CONDITION} edges, and each such edge could have a property \texttt{onset\_date} or \texttt{status} (active/resolved). In our extracted graph (which is de-identified and aggregated, not focused on individual patients), we likely have nodes representing medical concepts (like diseases, symptoms, medications) and edges representing the relations found in text (like "<Symptom> ASSOCIATED\_WITH <Disease>"). Neo4j's ability to handle property graphs means we can store things like confidence scores or context (like which model extracted this relation) as properties on the edge, which is very useful for later analysis or filtering.

\item \textbf{Cypher Query Language:} Neo4j uses \textbf{Cypher}, a declarative query language specifically designed for querying graphs. Cypher's syntax is often described as "SQL for graphs" but with pattern matching. For instance, a Cypher query to find what diseases a drug treats might look like:

\begin{verbatim}
MATCH (d:Drug {name:"Aspirin"})-[:TREATS]->(c:Disease)
RETURN c.name;
\end{verbatim}

This would find all \texttt{:Disease} nodes connected via a \texttt{TREATS} relationship from the \texttt{:Drug} node named "Aspirin", and return the disease names. Cypher uses ASCII-art style notation, where parentheses \texttt{()} denote nodes and \texttt{-[]->} denotes directed relationships. One can put conditions inside, like \texttt{\{property:value\}} or add variables. It's quite expressive – you can do aggregations, ordering, etc., similar to SQL, but the power is in pattern matching across arbitrary path lengths. For example, finding if there's any connection between two concepts within 3 hops is a simple Cypher query with path patterns. In our project, we generate Cypher queries programmatically to insert the data. For each extracted entity, we may do a \texttt{MERGE (n:Condition \{cui:"C0012345", name:"Diabetes"\})} to create or find a node, and for each relation, a \texttt{MERGE} for the edge like

\begin{verbatim}
MERGE (d:Drug \{cui:"C0004057"\})-[:TREATS \{source:"note1"\}]->
(c:Condition \{cui:"C0011849"\})
\end{verbatim}

Using \texttt{MERGE} ensures we don't create duplicates if the node already exists. We also may use Cypher for querying the graph once built, to answer questions like "how many unique medications were linked to each condition," or "find an example subgraph of a patient's note information." Cypher's readability (almost like describing the pattern you want) makes it easier for us to verify that the graph is constructed correctly.

\item \textbf{Advantages of Neo4j:} Apart from the model and query language, Neo4j is optimized for \textbf{graph traversals}. Queries that chase pointers through data (which in SQL would be multiple JOINs) are very fast in Neo4j even as the graph grows, because of index-free adjacency (each node directly knows its connected nodes). Neo4j also has features like full-text indexing (if we want to index node names for quick lookup), and APOC (a library of procedures) which can do things like computing shortest paths or other graph algorithms. There's also a web-based browser interface that lets you visualize query results, which is great for exploring the knowledge graph manually – you can literally see nodes and relationships plotted, which can reveal patterns or errors (like if an entity wasn't linked properly, it might appear isolated).

\item \textbf{Cypher Query Generation:} A part of our methodology (Chapter 3) is dedicated to how we translate extracted information into Cypher queries (Section 3.6.1). In practice, for each clinical note processed, our pipeline will output a series of Cypher commands to build that portion of the graph. We ensure queries are batched and transactions are used to make insertion efficient. The result is a Neo4j database containing the consolidated knowledge graph from all notes.
\end{itemize}

In summary, \textbf{Neo4j} provides a robust platform for our knowledge graph, and \textbf{Cypher} allows us to interact with that graph in a flexible way – whether to populate it or to retrieve insights. Many published clinical knowledge graph projects (and industry solutions) use Neo4j due to these strengths \cite{Zimbres2024}. By using Neo4j, we align with best practices and ensure that our graph is not just a conceptual outcome, but a queryable database that can answer complex questions about the data model we've constructed.

\subsection{Clinical Knowledge Graph Applications}

Knowledge graphs (KGs) in the clinical domain unlock a range of powerful applications by enabling connections across different types of medical data and knowledge. By converting unstructured text into a graph of entities and relationships, we create a structure that can be traversed and analyzed to support both clinical care and research. Here we review some key applications of clinical knowledge graphs, as reported in literature and practice \cite{Milvus2025}:

\begin{itemize}
\item \textbf{Integrating Patient Data for Holistic View:} One immediate use of KGs is to break down data silos in healthcare. Patient information is often fragmented across EHR free-text notes, structured tables, lab systems, etc. A knowledge graph can serve as a unifying layer by representing a patient and all related entities (conditions, drugs, lab results, procedures, family history, etc.) as a connected subgraph. For example, a KG might have a node for a patient that links to nodes for each diagnosis they have, each medication they are taking, and key findings from their last visit. Even without identifying the patient (for privacy, one could use an anonymous patient ID), this structured representation allows complex queries. A clinician or system could query: "Find all patients who have Diabetes and are on Drug X and have lab result Y > threshold," which is essentially a graph pattern search. Or in a single patient context: "Show me all known allergies and medications for this patient and any documented interactions." The graph naturally brings these pieces together. A published example of this concept is the use of KGs for a \textbf{Patient Timeline} or summary: by linking problems to treatments and outcomes over time, one can generate a chronological story of care. This integration is especially important for chronic diseases with long histories and multiple care providers. In a graph, if a patient's node is connected to a concept node "Type 2 Diabetes", and that concept node might connect further to a "Diabetes Mellitus" guideline node (as knowledge), one could quickly see if the patient's care aligns with recommended care (because the guideline node might connect to a recommended medication node, etc.). In our work, while we do not maintain patient-specific graphs (due to working on de-identified aggregated text), the principles learned can apply to patient-centric KGs in operational settings.

\item \textbf{Clinical Decision Support (CDS):} Knowledge graphs can enhance decision support systems by encoding medical knowledge and enabling reasoning over it. For instance, consider a KG that includes not just patient data but also nodes and edges representing clinical guidelines or best practices. One could model a rule like "Patients with Condition X should receive Test Y every 6 months" as a relationship in the graph between the Condition node and a Procedure node (perhaps with an edge type like "REQUIRES\_TEST"). Then, by linking patient nodes to their condition nodes and seeing if the "REQUIRES\_TEST" edge is followed by a "HAS\_DONE\_TEST" edge (for that patient), you can query for patients who are missing a recommended test. Another scenario is a graph that connects drug nodes to condition nodes they treat and contraindication nodes they conflict with. A CDS system could traverse the graph: given a patient's conditions and current medications (all nodes connected to that patient), are there any contraindication edges among those? If yes, alert the clinician. Some research prototypes have used knowledge graphs to do things like suggest diagnoses: they take the patient's symptoms and existing conditions as a subgraph, then look for a disease node that is connected to many of those findings (using the graph of medical knowledge that symptoms link to diseases) – essentially performing a reasoning that a doctor might do mentally. Graph queries (via Cypher or SPARQL) can implement such logic in a straightforward way by pattern matching. For example, find a disease node that has edges to all of a set of symptom nodes that the patient has, rank by how many of the patient's findings connect, etc. This could be used to generate differential diagnoses or catch something that might have been overlooked. In sum, KGs allow \textbf{automated inference} (or at least advanced querying) that goes beyond looking at each piece of data in isolation, thereby supporting clinical decisions with a more holistic consideration of relationships.
\end{itemize}

\section{Related Work Summary and Research Gap}

In this chapter, we reviewed the literature across the spectrum of our task: from natural language processing in the clinical domain (entity recognition and linking) to relationship extraction and knowledge graph construction. The existing body of work demonstrates substantial progress in each of these individual areas:

\begin{itemize}
\item \textbf{Clinical NLP Maturity:} The evolution from rule-based systems to statistical ML, and now to deep learning and large language models, has greatly enhanced the accuracy of extracting information from clinical text. There are robust tools and models for medical NER (e.g., BiLSTM-CRF models, BioBERT-based taggers) that can identify a wide range of entity types with high performance. Similarly, relation extraction techniques have evolved to capture increasingly complex relationships, especially with transformers and prompt-based LLM methods leading recent benchmarks. The literature shows that incorporating domain-specific knowledge (through biomedical embeddings or training on clinical corpora) consistently yields better results than purely general models. For example, domain-tuned models like ClinicalBERT outperform general BERT on tasks like i2b2 entity and relation extraction, highlighting the importance of medical context.

\item \textbf{Entity Linking and Knowledge Bases:} We saw that numerous approaches exist for linking entities to medical KBs, from straightforward dictionary methods (e.g., MetaMap's string matching) to advanced neural normalization algorithms. Knowledge bases like UMLS and MeSH provide the essential backbone of medical ontologies, and most works leverage these for standardization. One notable trend is that despite the emergence of sophisticated linking models, many practical systems still rely on or incorporate simple methods due to issues of efficiency and completeness. For instance, even state-of-the-art systems like BERN2 and PubTator (for biomedical text annotation) use hybrid approaches that include dictionary lookups for certain entity types to ensure high coverage. This indicates that purely end-to-end learned linking has not entirely supplanted older techniques, likely because of the challenges we discussed (ambiguity, scale, etc.). This informs our approach to use a proven tool (SciSpacy) and then build on it with custom logic, rather than trying to train a new linker from scratch.

\item \textbf{Knowledge Graph Construction \& Use:} There are multiple reports of building clinical knowledge graphs for different purposes, confirming the feasibility and value of this endeavor. For example, successful cases include constructing a medication interaction KG from electronic prescriptions, or a phenotype-genotype KG from medical literature, and using them to answer complex queries or predict outcomes \cite{Rotmensch2017}. A common theme is that combining data from different sources (EHR notes, structured clinical data, literature, guidelines) into a graph enables insights that siloed data cannot provide. However, each existing work tends to focus on a specific slice (e.g., a KG for cancer patient data, or a KG for pharmacovigilance). There isn't a one-size-fits-all clinical KG, and indeed our thesis is carving out its own niche: generating a knowledge graph from unstructured text with a \textbf{multi-model pipeline} and comparing general vs specialized models within that.
\end{itemize}

After surveying the literature, we identify two key gaps that motivate our research:

1. \textbf{Lack of a Unified Multi-Model Pipeline Evaluation:} Many studies address one piece of the puzzle (NER, linking, RE, or KG construction) in isolation. For instance, a paper might present a new NER model, or an improved relation classifier, evaluated on its own. In building an end-to-end system that goes from raw text to a knowledge graph, one often has to stitch together multiple components, and the interactions between those components can be complex. There is a need for exploring a \textbf{combined approach} where each task is handled by a model best suited for it (a "multi-model" approach), and evaluating the pipeline as a whole on real clinical text. Our thesis addresses this by integrating a domain-specific NER (GLiNER or similar) with a proven linker (SciSpacy/UMLS) and a large language model for relation extraction, thereby leveraging strengths of each. To our knowledge, prior work has not thoroughly documented or evaluated such a pipeline. For example, while one study might use SciSpacy for NER+linking and another might use BERT for end-to-end extraction, we are explicitly combining specialist tools (which is often how a practical system would be built) and will analyze how they complement or hinder each other. This includes how errors propagate (e.g., if NER misses an entity, no relation can be extracted for it) and how using multiple threads (parallel processing) impacts throughput. By fully \textbf{revising and integrating} the pieces, our work aims to provide a blueprint and assessment for multi-model information extraction pipelines in healthcare, which is an area not comprehensively covered in literature.

2. \textbf{Limited Comparative Analysis of General vs. Domain-Specific Models:} Another gap is the understanding of trade-offs between general NLP models and domain-specific models in the context of clinical IE tasks. While it's generally accepted that domain-tuned models outperform on in-domain evaluations, this has usually been shown on individual tasks like NER or classification. With the advent of very large general models (e.g., GPT-4) that have surprisingly strong capabilities even on biomedical text, the question arises: could a general model fine-tuned or prompted appropriately perform as well as (or even better than) a domain-specific model on clinical relation extraction and knowledge graph construction? And what about the practical considerations such as speed, cost, or data privacy? The literature so far lacks direct head-to-head comparisons in a full information extraction context. We intend to fill this gap by comparing "MedGemma3-4b" (a model from Google finetuned on de-identified medical text and images) versus "Gemma3-4b" (an instruction tuned version of the base model) on the task of extracting relationships that will form our knowledge graph. This comparative study will shed light on questions like: Does the medical model truly capture subtle clinical context better (e.g., understanding negations or the significance of certain lab findings), or can the general model make up for it with sheer parametric knowledge? Are there particular relation types where one class of model excels over the other? For instance, perhaps the general model, having seen open-text, might better handle unconventional phrasing, whereas the medical model might do better on medication dosages and specific jargon. No prior study, to our knowledge, has provided a systematic evaluation of this within one pipeline, which is what Chapter 4 of this thesis contributes.

In addition to these two major gaps, our work also pushes into some relatively under-explored areas such as the practical performance of running such pipelines on modern hardware (e.g., how to optimize a multi-model pipeline on limited memory Apple Silicon, as noted in Chapter 5) which we consider important for real-world adoption.

By addressing the above gaps, the thesis aims to advance the state-of-the-art in transforming unstructured clinical text into actionable knowledge graphs. We not only propose a novel pipeline architecture but also critically evaluate the components and alternatives. The end result will be a clearer understanding of how to best combine tools and models for clinical information extraction, and evidence-based insight into the value of domain-specific versus general AI models in this setting. This bridges the isolated advances in literature into an integrated solution, moving a step closer to practical deployment in healthcare environments.
 % Literature Review and Theoretical Background
% Chapter 3

\chapter{Methodology} % Main chapter title

\label{Chapter3} % For referencing the chapter elsewhere, use \ref{Chapter3}

%----------------------------------------------------------------------------------------

This chapter details the multi-model methodology for transforming unstructured clinical notes into a structured medical knowledge graph. The overall approach is a \textbf{pipeline} that sequentially performs entity recognition, entity linking, relationship extraction, and graph construction. By chaining specialized components, the system produces an \textbf{interpretable representation of medical concepts (e.g. drugs, diseases) and the relations among them}, enabling integration of context and supporting clinical insights. Key design decisions, data preprocessing steps, model configurations, and implementation optimizations are discussed in the following sections.

\section{System Architecture and Pipeline}

\subsection{Overall System Design}

The proposed system follows a modular pipeline architecture, where each stage transforms the clinical text and feeds into the next stage. Figure~\ref{fig:pipeline} outlines the architecture with its major components and data flow:

\begin{enumerate}
\item \textbf{Input and Preprocessing:} Raw clinical notes (free-text) are first collected and preprocessed (cleaning, formatting) to ensure consistent input for NLP tasks.
\item \textbf{Entity Recognition:} A medical Named Entity Recognition (NER) model identifies spans of text corresponding to clinical concepts (e.g. diseases, medications, procedure, body site). We employ the GLiNER model for this step (see Section~\ref{sec:gliner}).
\item \textbf{Entity Linking:} Detected entity mentions are normalized and linked to a canonical identifier in a medical knowledge base. We use SciSpacy's UMLS linker to map each mention to a UMLS Concept Unique Identifier (CUI), providing standardized meanings (Section~\ref{sec:entitylinking}).
\item \textbf{Relationship Extraction:} A large language model (LLM) processes the linked entity list and full text to extract relationships between the medical entities. The model outputs structured triple relationship assertions (Section~\ref{sec:relationextraction}).
\item \textbf{Knowledge Graph Construction:} The extracted entities and relationships are inserted into a Neo4j graph database. Nodes represent medical concepts (annotated with UMLS CUIs and types) and edges represent the relationships between concepts. Cypher queries are generated to create or merge these graph elements (Section~\ref{sec:graphconstruction}).
\item \textbf{Post-processing \& Optimization:} Throughout the pipeline, confidence filters, validation checks, and parallelization are applied to ensure quality and efficiency (Section~\ref{sec:optimization}).
\end{enumerate}

Each component addresses a specific task, and their integration yields a coherent end-to-end system. This design allows independent optimization of each module and the ability to swap models if needed. The multi-step \textbf{``recognition–linking–relation extraction–graph loading''} strategy aligns with recommended knowledge graph construction practices, ensuring that unstructured text is incrementally converted into a structured, interoperable knowledge graph.

\subsection{Component Integration Flow}

The flow of data between components is carefully orchestrated to preserve context and accuracy. The pipeline is implemented as a \textit{sequential workflow} where each module consumes the outputs of previous steps:

\begin{itemize}
\item The \textbf{Preprocessing} stage standardizes the input. This may include lowercasing (if required by the NER model), removing extraneous whitespace or headers, and ensuring patient-identifying information is excluded for privacy (see Section~\ref{sec:datapreprocessing}).
\item The \textbf{NER component} (GLiNER) is applied next on the cleaned text. It produces a set of entity mentions (text spans) each with an entity type (e.g. \textit{Condition}, \textit{Drug}, \textit{Test}). These mentions are passed forward as Python objects (e.g. spaCy \texttt{Span} objects) attached to the document representation.
\item The \textbf{Entity Linking} module (SciSpacy's \texttt{EntityLinker}) takes each NER-detected span and searches for candidate concepts in the UMLS knowledge base. Because this linker is integrated as a spaCy pipeline component, it can enrich the detected \texttt{Span} with a list of candidate CUIs and similarity scores (available via \texttt{Span.\_.kb\_ents}). We configure the linker to resolve abbreviations and to attach the top-ranked CUI for each mention above a confidence threshold (details in Section~\ref{sec:confidence}). The result is that each entity mention from NER is now linked to a unique medical concept ID or marked as \textit{unlinkable} if no high-confidence match is found.
\item For \textbf{Relationship Extraction}, the entire clinical note along with the recognized entities is provided to the LLM. In practice, we construct a prompt that includes the text and possibly the list of identified entities, asking the model to output relationships (e.g. cause/effect, treatment, associations) among those entities. The integration is such that the model's output can be traced back to the original entities by name or CUI. This output is captured as structured relation triples.
\item Finally, the \textbf{Graph Construction} module translates the entities and relations into Cypher queries. The integration here uses the results of previous steps (entity CUIs, names, types, and relation types) to form \texttt{MERGE} statements. For example, if ``C0011849'' (Diabetes mellitus) and ``C0027796'' (Neuropathy) are two CUIs with a \textit{causes} relationship, the pipeline will generate a Cypher command to \texttt{MERGE} nodes for each (with labels/types) and a \texttt{[:CAUSES]} relationship between them. These queries are executed on the Neo4j database to update the knowledge graph.
\end{itemize}

Throughout this flow, \textbf{state is maintained} so that each mention's context is known. For instance, if an LLM-produced relationship mentions an entity not recognized by NER, we detect that during parsing and handle it (either by discarding that relation or attempting to link the new entity, as discussed in Section~\ref{sec:outputparsing}). This ensures consistency: only vetted entities make it into the graph. The sequential integration also means errors can propagate (e.g. a linking error could affect relation extraction), so each component is tuned to maximize precision, and wherever possible, the pipeline includes verification steps (for example, ensuring that relation arguments have valid CUIs). The design balances modularity with tight integration, creating a robust system to extract structured knowledge from clinical text.

\section{Data and Preprocessing}
\label{sec:datapreprocessing}

\subsection{Clinical Notes Dataset and Preparation}

Our dataset consists of unstructured clinical notes that include various document types such as discharge summaries, progress notes, and radiology reports. These notes are written in natural language by healthcare professionals and often contain domain-specific terminology, abbreviations, and sensitive information. Before applying the NLP pipeline, we perform several preprocessing steps to prepare this data:

\begin{itemize}
\item \textbf{De-identification:} To address privacy and ethical considerations, all Protected Health Information (PHI) in the notes is removed or obfuscated. This includes patient names, IDs, dates, and locations (in accordance with HIPAA guidelines). In practice, we either use an automated de-identification tool or rely on the dataset being pre-deidentified by its provider (as is common with public clinical corpora).
\item \textbf{Format Normalization:} The notes are normalized for consistent formatting. For example, irregular whitespace, line breaks, or punctuation are standardized. Sections like ``Patient History:'' or template headers might be stripped out if not relevant, as they can confuse the NER model. We ensure the text encoding is uniform (UTF-8) and fix any encoding-related artifacts.
\item \textbf{Spelling and Abbreviation Expansion:} Clinical notes often contain domain-specific shorthand (e.g., ``HTN'' for hypertension). While our pipeline has a dedicated abbreviation detection step (Section~\ref{sec:confidence}), in preprocessing we compile a lexicon of common shorthand terms and ensure they can be recognized. We do not perform aggressive spelling correction, since misspellings might be rare or could unintentionally alter clinical meaning, but obvious OCR errors or gibberish are cleaned if encountered.
\end{itemize}

After preprocessing, clinical notes are clean and standardized while preserving clinical content integrity. This careful preparation facilitates better entity recognition and linking by ensuring text resembles model training forms and helps avoid cascading errors through consistent abbreviation detection. These steps address the \textbf{``garbage in, garbage out''} concern - improving input quality enhances extracted knowledge reliability.

\section{Entity Recognition and Linking}

\subsection{GLiNER Model for Medical Entity Recognition}
\label{sec:gliner}

For identifying medical entities in text, we utilize the \textbf{GLiNER} model – a \textit{Generalist and Lightweight Named Entity Recognition} framework adapted for biomedical data. Traditional medical NER is challenging due to vast evolving healthcare vocabulary and limited labeled training data. GLiNER offers \textbf{open-domain, zero-shot NER} using \textit{natural language descriptors} rather than fixed entity taxonomies. We can prompt GLiNER with definitions like medical condition'' or drug name,'' and it identifies corresponding text spans.

GLiNER-BioMed leverages large language models for annotation, then distills knowledge into smaller, efficient NER models. Developers generated synthetic training data using LLMs, then trained GLiNER models on this data. The result achieves state-of-the-art biomedical NER performance with 6% F1-score improvement over previous systems in zero-shot scenarios, statistically significant at p < 0.001 \parencite{Stenetorp2024}.

In our pipeline, we chose GLiNER for its generalization ability, configuring it with \textbf{medical entity type prompts}: \textit{Problem/Condition}, \textit{Treatment/Drug}, \textit{Test/Procedure}, and others as needed. GLiNER can tag entities without dataset-specific retraining. For \textit{The patient was started on metformin for diabetes mellitus''}, GLiNER identifies \textbf{metformin''} as Drug and \textbf{``diabetes mellitus''} as Condition based on prompt understanding.

GLiNER's \textbf{lightweight architecture} is smaller and faster than full LLMs, suitable for scanning long documents. It effectively \textit{distills} larger models' linguistic knowledge into specialized NER tasks, running efficiently on available compute (Apple Silicon CPU/GPU) with reasonable speed crucial for clinical text processing.

\subsection{UMLS Entity Linking with SciSpacy}
\label{sec:entitylinking}

Once entities are recognized, we perform \textbf{entity linking} to anchor mentions to standardized medical ontology using SciSpacy's UMLS \textbf{EntityLinker} for biomedical normalization. SciSpacy provides spaCy models tailored to scientific/clinical text \parencite{Neumann2019}, with entity linker containing a built-in knowledge base from the \textbf{Unified Medical Language System (UMLS)} Metathesaurus integrating millions of biomedical concepts.

SciSpacy's linking algorithm uses \textbf{string similarity} with \textit{character n-grams}, representing entity mentions as bag-of-character-trigrams with \textbf{TF-IDF weighted vectors}. Every UMLS concept name is indexed by character trigrams, enabling approximate nearest neighbor search for most similar concepts. This finds overlapping substrings - robust for biomedical text with lexical variants, like cardiac infarction'' matching \textit{Myocardial Infarction''} through trigram overlap.

We configured SciSpacy with \texttt{\detokenize{linker_name="umls"}} and \texttt{\detokenize{resolve_abbreviations=True}} to expand short forms before linking. The UMLS knowledge base includes primary vocabularies totaling roughly 3 million concepts. We kept the default \textbf{similarity threshold} of 0.7, meaning linkers only assign concepts when cosine similarity between TF-IDF trigram vectors is $\geq 0.7$, filtering tenuous matches and improving precision.

During linking, SciSpacy produces candidate concept IDs with similarity scores for each entity. We select top-ranked candidates exceeding the threshold. For metformin,'' the linker returns (CUI: C0025598, Metformin'', score 0.98), providing \textit{canonical representation} with UMLS Concept Unique Identifiers and standardized names, definitions, and semantic types.

Linking to UMLS creates \textbf{interoperable knowledge graphs}. Grounding entities to UMLS CUIs aligns nodes with established ontology, facilitating healthcare data integration and hierarchical queries. It consolidates synonyms - heart attack'' and myocardial infarction'' link to the same CUI, creating single graph nodes and avoiding duplication \parencite{UMLS2024}.

\subsection{Confidence Scoring and Abbreviation Detection}
\label{sec:confidence}

In this stage, we refine the outputs of NER and linking by incorporating confidence measures and handling abbreviations explicitly:

\begin{itemize}
\item \textbf{Confidence Scoring:} Both NER and linker provide internal scores. After entity linking, we use SciSpacy's \textbf{similarity score} as confidence proxy. Poor matches (0.4 similarity) may be dropped, while high scores (close to 1.0) indicate confident matches. We examine score distributions - if best match is 0.72 and second is 0.71, that's ambiguous; if best is 0.85 and second is 0.60, that's clear-cut. We set thresholds: below 0.7 marks entities as "Unlinked," above 0.9 accepts outright, and 0.7-0.9 accepts with lower confidence flags. This approach maintains knowledge graph quality by minimizing spurious nodes/edges.

\item \textbf{Abbreviation Detection:} Clinical notes contain numerous abbreviations requiring resolution for correct linking. We integrate SciSpacy's \textbf{AbbreviationDetector} implementing Schwartz \& Hearst algorithm \parencite{Schwartz2003}, scanning for patterns like ``Full Form (Abbr).'' For \textit{``Patient has chronic obstructive pulmonary disease (COPD)''}, it detects \textbf{``COPD''} abbreviates \textbf{``chronic obstructive pulmonary disease.''} With \texttt{resolve\_abbreviations=True}, the EntityLinker uses long forms for UMLS matching, yielding correct high-confidence links like CUI C0019061. This particularly helps ambiguous abbreviations like ``RA'' by using local context for expansion.
\end{itemize}

By applying confidence scoring and abbreviation resolution, we \textbf{improve the precision and recall} of the entity linking process. High-confidence links and fully expanded terms result in more correct nodes in the graph, and fewer missed entities. These measures also reduce noise for the next stage: the relationship extraction model will receive text where abbreviations are already expanded (in the \texttt{Doc} object's context) and where uncertain entities can be treated cautiously. Overall, this step solidifies the foundation of the knowledge graph by ensuring that we have \textit{trusted, well-defined entities} to work with.

\section{Relationship Extraction}
\label{sec:relationextraction}

\subsection{Model Selection and Prompt Engineering}

Extracting relationships from clinical text is a complex task, as it involves understanding the semantic connections between medical entities. We approach this with a \textbf{Large Language Model (LLM)} that can interpret the text's meaning and generate relational triples. We consider two LLMs for this task: \textbf{Gemma} and \textbf{MedGemma}. Gemma is a general-purpose language model (built on the Gemma 3 architecture), whereas MedGemma is a specialized variant fine-tuned for medical domain knowledge. MedGemma was developed by Google DeepMind in 2025 as an open model for medical text and has demonstrated advanced medical understanding and clinical reasoning capabilities. Essentially, MedGemma builds on the strengths of Gemma but with domain-specific training, making it more adept at understanding clinical context and terminology \parencite{MedGemma2025,Gemma2025}.

Given the critical nature of accurate relation extraction, we opted to experiment with both models. MedGemma (specifically the 4B variant) is expected to have an edge in capturing medical relationships (e.g., drug–disease interactions, symptom–disease associations), whereas Gemma (general model) provides a baseline to see how a non-medical-tuned model performs on the same task. This comparative aspect will be explored in Chapter 4, but for the methodology, the pipeline is designed to be \textit{model-agnostic} – it can plug in any LLM that accepts a prompt and returns text.

We use a \textbf{prompt-based approach} (in-context learning) rather than fine-tuning, to leverage these models directly on our extraction task. The prompt is carefully engineered to guide the model to output the information in a structured format. Based on best practices from recent studies, a good prompt clearly defines the task, provides examples, and indicates the desired output format \parencite{Reynolds2021}. Our prompt typically consists of instructions like:

\begin{itemize}
\item A brief task description: e.g., \textit{``Extract all clinically relevant relationships between medical concepts in the following text.''}
\item A format instruction: e.g., \textit{``Provide the relationships as a list of triples (Subject, Relation, Object) using the exact entity names from the text.''} We explicitly ask for the model to use the entities as mentioned, to ease alignment with CUIs.
\item Optionally, a few-shot example: for instance, showing the model an example sentence and the extracted triple from it. We might include one or two demonstration pairs if it improves performance, although with very large models often a clear instruction suffices.
\item The context text: the actual clinical note or a segment of it, possibly truncated to stay within token limits of the model.
\end{itemize}

An example prompt might be:

\begin{verbatim}
Extract all relationships between medical entities in the text. 
Use the format (Entity1, Relation, Entity2).
Text: "The patient's diabetes caused peripheral neuropathy and 
he was prescribed gabapentin for pain management."
Relationships:
1. (diabetes, causes, peripheral neuropathy)
2. (gabapentin, treats, pain)
\end{verbatim}

In this prompt, we provided a made-up example demonstrating the expected output format. The actual note's text would follow after ``Text:'' and we would expect the model's completion to list similar triples.

We also incorporate prompt elements to handle nuances: e.g., instruct the model to ignore trivial relations or to only output relations that are explicitly or implicitly stated (to avoid hallucination). We emphasize that the output should not include any entity not found in the text. This is important because LLMs have a tendency to infer or hallucinate facts; by explicitly saying \textit{``use only entities from the text''}, we reduce the chance the model introduces an unrelated concept.

To summarize, our prompt engineering strategy focuses on clarity, examples, and format enforcement. We aim to push the model to behave almost like a rule-based extractor but backed by its deep understanding of language. This harnesses the best of both worlds: the model's intelligence and a deterministic output scheme. We will quantitatively compare the two chosen models' outputs later, but here it's worth noting that using an LLM for relation extraction aligns with the latest research trends in knowledge graph construction. Large foundation models have been successfully used to perform relation extraction without extensive task-specific training, by virtue of their pre-trained knowledge and language understanding \parencite{Singhal2022}. Our methodology capitalizes on this capability by employing prompt-based LLM queries to extract rich relational information from text that simpler models or rule-based systems might miss.

\subsection{Output Parsing and Validation}
\label{sec:outputparsing}

After the LLM produces candidate relationships in text form, we need to parse these outputs and validate them before integration into the knowledge graph. This step is critical for maintaining accuracy and consistency, as the raw model output may contain noise or require interpretation.

\begin{itemize}
\item \textbf{Output Parsing:} Given that we instruct the model to output triples in a structured format (as in the example list format or a JSON), the parser's job is to interpret that format. In many cases, the model's response can be read line by line: each line containing a triple like \texttt{(Entity1, relation, Entity2)}. We implement a parser that uses regex or string splitting to extract the three components of each triple. For example, from a line ``\texttt{(diabetes, causes, peripheral neuropathy)}'', the parser will strip the parentheses and split by comma, yielding subject = ``diabetes'', relation = ``causes'', object = ``peripheral neuropathy''. We then trim whitespace and ensure the text exactly matches something in the original note. If the model returns a JSON structure (less likely in our case, since we kept it simple), we would use a JSON parser to get the fields.

\item \textbf{Entity Matching:} We cross-match the extracted entity text with the list of recognized entities from the NER stage. This is where having the exact text form is important. Ideally, since we told the model to use exact names from text, each Entity1 and Entity2 from the triple should correspond to an NER-identified mention. We do a lookup: for each parsed entity string, find the entity in the document (we can utilize character offsets or a direct string match in the note). If an entity from the model output is not found in the original text or wasn't tagged by NER, this raises a red flag. In such cases, we have a few strategies:

  \begin{itemize}
  \item If it's a minor variation (e.g. model output ``diabetes'' but text had ``diabetes mellitus''), we can potentially still match it by substring or using the linked CUI (since both share a CUI). Our pipeline can be smart here by checking the UMLS synonyms – \textit{diabetes} as a lay term vs \textit{diabetes mellitus} as formal term are the same concept. We might normalize both and realize they match.
  \item If it's something entirely not in text, it's likely a hallucination or an inference beyond instruction. We then \textbf{discard that triple}. For example, if the model output a triple with ``insulin'' as an entity but the note never mentioned insulin, this would be dropped. Such validation is supported by known methods; researchers have even used \textit{logit biasing} to prevent models from generating tokens not in the source text. While we did not fine-tune at that level, our post-hoc validation essentially achieves the same outcome by filtering out any extraneous tokens/entities \parencite{Ji2023}.
  \end{itemize}

\item \textbf{Relation Validation:} We also validate the relation part of each triple. We check that the relation is expressed in a reasonable way (usually a verb or a short phrase). In clinical context, relations might be ``causes,'' ``treats,'' ``indicates,'' ``associates with,'' etc. If the model somehow produced a very lengthy or dubious relation phrase, we could flag it. Additionally, we verify that the relation is not obviously contradicted by context (though automated contradiction detection is hard). At minimum, we ensure the relation is not empty and not identical to the entities (sometimes a model might output something malformed like \texttt{(X, X, Y)} which doesn't make sense, and that would be removed).

\item \textbf{Duplication and Uniqueness:} The parser will also handle duplicates. The LLM might list the same relationship twice or express it in two similar ways. We canonicalize the triple (e.g. by sorting or lowercasing the relation phrase) and use a set to ensure uniqueness. Only unique triples proceed to the graph insertion stage. If the relation is symmetric or if order doesn't matter (e.g. co-occurrence), we would handle that, but in our case most relations are directed (e.g. cause vs caused-by). We treat \texttt{(diabetes, causes, neuropathy)} as distinct from \texttt{(neuropathy, causes, diabetes)} and rely on the model to preserve the correct direction.

\item \textbf{Confidence and Post-Filtering:} Each triple inherits a confidence from the process. Since the LLM doesn't give an explicit probability per relation, we estimate confidence based on the model's overall reliability and the presence of certain keywords. For instance, if the relation phrase was directly taken from the text (e.g. ``caused''), we have high confidence the relation is correctly identified. If the model had to infer something not explicitly stated, confidence is lower. In experiments, we found that MedGemma tends to stick closer to text, whereas a general model might infer more. We thus weight triples from MedGemma slightly higher in confidence by default (this will be quantitatively examined in evaluation). In any case, we allow all extracted triples above a certain basic confidence to be inserted into the graph, but we tag those that are iffy (for potential exclusion in analysis).
\end{itemize}

As a result of parsing and validation, we obtain a \textbf{clean set of (Subject, Relation, Object)} triples, each linked to UMLS concept IDs via the subject and object. For example, ``(diabetes mellitus, causes, peripheral neuropathy)'' becomes (CUI C0011849, \textit{causes}, CUI C0031117) after we replace the text with the corresponding CUIs from the earlier linking stage. These are now ready to be ingested into the knowledge graph. The validation steps, while reducing quantity slightly, ensure that the \textit{quality of relationships is high} – only those supported by the text and recognized entities are kept. This mitigates the risk of hallucinations or errors from the LLM contaminating the knowledge base. By designing the prompt and parser in tandem, we effectively constrain the LLM's output and then double-check it, achieving a balance between \textbf{completeness and accuracy} in relationship extraction.

\section{Knowledge Graph Construction}
\label{sec:graphconstruction}

\subsection{Graph Schema Design}

With a list of extracted entities (linked to UMLS) and relations, we define a schema for how these should be represented in the knowledge graph. The schema outlines what node types and relationship types exist, and what properties they carry, ensuring the graph is both expressive and normalized.

\begin{itemize}
\item \textbf{Node Types and Properties:} In our design, each \textbf{medical entity} becomes a node in the graph. Rather than having dozens of node categories for each fine-grained entity type, we opt for a unified node label, say \textbf{\texttt{:MedicalEntity}}, with properties to encode its attributes. Every node has at least:

  \begin{itemize}
  \item \textbf{\texttt{cui}}: the UMLS Concept Unique Identifier (as a string). This is the primary key for the node; we assume one node per unique CUI. Using CUI ensures that if the same concept appears in multiple notes, they map to the same node.
  \item \textbf{\texttt{name}}: a canonical name for the concept. We use the preferred name from UMLS for that CUI (e.g., C0011849 → ``Diabetes Mellitus''). This makes the graph human-readable.
  \item \textbf{\texttt{sem\_type}}: the semantic type or category of the concept, as defined in UMLS (e.g., T047 for ``Disease or Syndrome'', T121 for ``Pharmacologic Substance''). We might also include a more readable form of the semantic type (like a label ``Disease''). This property allows filtering or subtyping of nodes by broad category (e.g., query all nodes that are diseases).
  \item Optionally, we store other metadata if available: e.g., \textbf{\texttt{definition}} (a short definition from UMLS), \textbf{\texttt{aliases}} (synonyms), etc. We did not fully exploit these in our core pipeline due to size concerns, but including definitions could help if one were to use the graph for reasoning or verification.
  \end{itemize}

  We did consider splitting node labels by major semantic group (e.g. \texttt{:Disease}, \texttt{:Drug}, \texttt{:Procedure} labels). This would mirror domain ontology classes and perhaps optimize certain queries. However, UMLS semantic types are numerous, and a node can have multiple semantic types. For simplicity, we use a single label and rely on the \texttt{sem\_type} property for distinguishing types. This decision keeps the schema flexible and avoids schema changes when new types appear.

\item \textbf{Relationship Types:} Each extracted relation becomes an edge in the graph connecting two MedicalEntity nodes. We capitalize and possibly normalize the relation phrase to define the \textbf{relationship type}. For example, ``causes'' may become a \texttt{:CAUSES} relationship type in Neo4j. If a relation phrase is longer (more than one word), we either use it verbatim (spaces are allowed in Cypher if quoted) or convert to CamelCase or snake\_case (e.g. ``side effect of'' could be stored as \texttt{:SIDE\_EFFECT\_OF}). We compiled a list of relation types we expect from our model outputs. Many will be symmetric inverses (e.g. ``treats'' vs ``treated\_by''). We decided to store only one direction as the relation type that was extracted. In cases where the inverse makes sense, Neo4j can query the reverse direction without duplicating edges (by traversing in reverse). For instance, if we have \texttt{(:Drug)-[:TREATS]->(:Disease)}, we don't need a separate \texttt{:TREATED\_BY} edge, as the inverse can be inferred in queries. Thus, our relationship types are directed as per the text's implication. Some examples of relationship types in our schema:

  \begin{itemize}
  \item \texttt{CAUSES} – from condition to outcome (disease to complication).
  \item \texttt{TREATS} – drug or procedure to condition.
  \item \texttt{ASSOCIATED\_WITH} – a general link if model says ``X is associated with Y''.
  \item \texttt{INDICATES} – symptom or test result indicating a condition.
  \end{itemize}
  
  These are not fixed by the system initially; instead, they emerge from the model output. We then standardize them as needed. If synonyms appear (e.g. ``leads to'' vs ``causes''), we may choose to map them to one canonical relation type for consistency.

\item \textbf{Graph Orientation and Context:} Each edge could also carry a \textbf{\texttt{context}} or \textbf{\texttt{source}} property indicating from which note (or sentence) it was derived. This is valuable if we want to trace back the provenance of a relationship. In our implementation, we include a property \texttt{note\_id} or \texttt{sentence\_id} on the relationship to record this. For example, \texttt{(Diabetes)-[CAUSES \{source:``Note42''\}]->(Neuropathy)} tells us that the relation was found in Note 42. This is useful for downstream validation or if we want to retrieve the original evidence for a given edge. We also considered temporal context (if notes have timestamps and the relation is time-bound) but our data does not deeply explore temporality, so we left that out of scope.
\end{itemize}

The resulting schema is a \textbf{property graph} schema common in biomedical KGs: a single dominant node type for concepts connected by various relation types. This design aligns with other healthcare knowledge graphs (e.g., HetioNet or OpenBioLink) which also have nodes for entities and typed edges for relationships, albeit those are often predefined by ontologies \parencite{Himmelstein2017}. In our case, the schema is partly \textit{emergent} (relation types come from text) and partly \textit{ontological} (node identities come from UMLS). By mapping to UMLS, we ensure the graph can be merged or aligned with existing knowledge sources. If needed, one could enrich this graph by pulling in more UMLS connections (like hierarchical relations ``isa'' between concepts), but our focus is on the information extracted directly from clinical narratives \parencite{Cowell2020}.

In summary, the schema is designed to capture the essential pieces: \textbf{unique medical entities} (with their standard identifiers and categories) and \textbf{meaningful relationships} between them as observed in clinical text. It strikes a balance between specificity (not losing detail of relation phrases) and interoperability (using global IDs for concepts). This schema will support the queries and analyses described in later chapters, such as counting unique entities, listing all relations of a certain type, or measuring graph connectivity.

\subsection{Cypher Query Generation and Storage}

Once we have the schema and the list of nodes and relations (with their properties) ready, we proceed to create the knowledge graph in a Neo4j database. We use the Cypher query language for graph database operations, generating queries programmatically for each element. The process is as follows:

\begin{itemize}
\item \textbf{Node Merge Queries:} For each unique entity (identified by CUI) extracted from the notes, we generate a Cypher \texttt{MERGE} query to ensure a corresponding node exists in the database. We prefer \texttt{MERGE} over \texttt{CREATE} to avoid duplicate nodes, as the same entity may appear multiple times. For example, for a concept C0011849 (Diabetes Mellitus) with name ``Diabetes Mellitus'' and sem\_type ``Disease or Syndrome'', we produce:

\begin{verbatim}
MERGE (n:MedicalEntity {cui: "C0011849"})
  ON CREATE SET n.name = "Diabetes Mellitus", 
                n.sem_type = "Disease or Syndrome";
\end{verbatim}

  This query checks if a node with \texttt{cui} = C0011849 exists. If not, it creates one and sets its properties. If it exists, it leaves it as is (or we could optionally update the name to ensure consistency, but typically the first creation suffices). We do this for each distinct CUI from our list of entities. These queries are often batched for efficiency (we can combine multiple MERGEs in one transaction, or send them sequentially).

\item \textbf{Relationship Creation Queries:} For each relationship triple (with source CUI, relation type, target CUI), we generate a \texttt{MERGE} or \texttt{CREATE} query for the edge. We use \texttt{MERGE} similarly to avoid duplicating the same edge if processed twice. However, we also include uniqueness by context if needed. The pattern looks like:

\begin{verbatim}
MATCH (a:MedicalEntity {cui:"C0011849"}), 
      (b:MedicalEntity {cui:"C0031117"})
MERGE (a)-[r:CAUSES]->(b)
  ON CREATE SET r.source = "Note42"
\end{verbatim}

  This query first finds the two nodes by their CUIs (we assume they've been created by the previous step). Then it merges a \texttt{:CAUSES} relationship from the first to the second. If the relationship did not exist, it will be created and we set a property \texttt{source} to ``Note42'' (for example). If it already exists (meaning we encountered the same relation earlier, perhaps from another sentence or note), the \texttt{MERGE} will match it and do nothing else. One caveat: In Neo4j, \texttt{MERGE} on a relationship without specifying all properties could merge even if \textit{source} differs, which might or might not be desired. If we want multiple provenance, we might allow duplicates distinguished by \texttt{source}. In our case, we decide that the existence of an edge means the two concepts are related; we don't create duplicate edges for multiple notes, but we could append sources. (One could model sources as an array property or connect a \texttt{:Occurrence} node, but we keep it simple.)

\item \textbf{Graph Storage:} The Neo4j database stores the resulting knowledge graph persistently. We verify the storage by running sample queries. For instance, after insertion, a query like:

\begin{verbatim}
MATCH (n:MedicalEntity)-[r]->(m:MedicalEntity) 
RETURN n.name, type(r), m.name LIMIT 5;
\end{verbatim}

  would return some sample triples, confirming data presence. We also check that no unintended duplicates exist: e.g., each CUI yields exactly one node (we rely on the MERGE logic for that). The use of an \textbf{ACID transactional database} like Neo4j ensures that even if our batch insertion is interrupted, we won't end up with partial duplicates; the transactions handle consistency.
\end{itemize}

One benefit of using Neo4j and Cypher is the ability to later query complex patterns, such as \textit{``find all drugs that treat complications of diabetes''} in a relatively straightforward graph traversal query. The choice of Cypher is natural since Neo4j is one of the most popular graph databases, and it has been used in multiple healthcare knowledge graph projects for its robust query capabilities \parencite{Neo4j2023}. Neo4j's flexibility also means we could augment the graph with new node types or relationships easily if our schema evolves.

Finally, it's worth noting that we considered the alternative of using a RDF triple store or an RDF representation (with entities as subjects/objects and relations as predicates). We opted for Neo4j's labeled property graph model for simplicity and familiarity. Our Cypher generation approach is straightforward, but it could be automated with an Object-Graph-Mapper or by using libraries like Py2neo or Neo4j's bulk import tool if scaling up. In this thesis context, generating explicit Cypher statements gave us fine control and transparency over the insertion process, which was helpful in debugging and verifying each step of the pipeline.

By the end of this stage, the unstructured text has been fully transformed: we now have a \textbf{populated knowledge graph} where each node is a medical concept (with context from a clinical note) and each edge encodes a relationship that was described in the narrative. This graph is ready to be analyzed for insights, queried for specific patterns, and evaluated against our research questions (as will be done in subsequent chapters).

\section{Implementation Optimization}
\label{sec:optimization}

\subsection{Parallel Processing and Resource Management}

Building the pipeline in a Jupyter notebook environment (on Apple Silicon hardware) required careful optimization to handle the computational load of NLP tasks and large knowledge bases. We implemented several strategies to improve runtime and manage memory:

\begin{itemize}
\item \textbf{Parallel Processing:} Many steps of the pipeline can be executed in parallel, especially at the document level. Since processing each clinical note is mostly independent of others, we utilized parallelism to speed up the pipeline. In Python, this was achieved with multiprocessing pools or joblib to distribute the work across CPU cores. For example, we could spawn multiple worker processes, each handling a subset of the notes through the entire NER→Linking→Relation Extraction sequence. This led to near-linear scaling in throughput with the number of cores, up to the point where shared resource contention (like memory or disk I/O) kicked in. We had to be cautious with SciSpacy's UMLS linker in a multi-process setting; loading the 1GB knowledge base in each process is expensive. To mitigate that, we explored two approaches: (a) initialize the linker in the parent process and use threading (where memory is shared) instead of multiprocessing for that component, and (b) use fewer processes for the linking step while still parallelizing lightweight steps (like just NER). Ultimately, we found a balance by parallelizing at a coarse level – e.g., processing 3 notes concurrently – which improved speed without overwhelming the system \parencite{McKerns2021}.

\item \textbf{Batching of Model Inference:} The GLiNER NER model and the LLM for relation extraction can potentially be run on batches of text rather than one document at a time. We took advantage of this for NER: combining a few notes together (if short) into one batch for the model significantly increased GPU/accelerator utilization. The HuggingFace transformer backend for GLiNER supports batching, so we used a batch size of 8 sentences per inference call, for instance. This means GLiNER would tokenize and run those sentences together through the model, amortizing the overhead. We ensured that this batching did not mix content from different notes in a way that confuses the model (each sentence in the batch is handled independently by the model, which is fine for NER). For the LLM (MedGemma/Gemma), batching is trickier because the prompt+output for each note is quite large and models like these typically don't support multi-prompt batching easily. Instead, we processed each note's relations sequentially for the LLM, which is a bottleneck. In future, an idea would be to use a smaller relation extraction model that could batch process multiple sentences, but given our use of a powerful but heavy LLM, we treated it as one-note-at-a-time. To compensate, we parallelized LLM calls across processes if possible (with caution to not overload memory).

\item \textbf{Execution Mode – In-Notebook vs. External:} Running everything inside a Jupyter notebook has convenience but some limitations (as noted in Chapter 5.5). To optimize, one technique we used was to offload long-running tasks to external scripts when needed. For instance, heavy graph database insertion could be done via Neo4j's bulk import tool outside the notebook if performance became an issue. In our experiments, the dataset size was moderate enough to handle within the notebook by chunking work and using the above optimizations. We also scheduled certain long steps (like processing with the LLM) to run and save outputs (like triples) to disk so that if the kernel died or restarted (common with large memory tasks), we wouldn't lose all progress.
\end{itemize}

In essence, through parallelism we improved speed, and through careful memory management we averted crashes and slowdowns. The result is a pipeline that, while computationally intensive, is feasible to run in a research setting. For example, if a single note took $\sim$30 seconds for the LLM to process, processing 100 notes sequentially would be $\sim$50 minutes, but with 4 parallel workers, we brought that down significantly (plus GLiNER and linking are much faster, so the LLM was indeed the slowest part). By keeping an eye on resource usage (memory/CPU cores) and optimizing where possible, we ensure that the methodology is not just theoretically sound but also practically executable within our hardware and time constraints.

Overall, the methodology chapter has described how each component of our system works and how they come together efficiently. In the subsequent chapters, we will assess how well this methodology performs, our results by comparing the chosen models, and discuss the implications of this approach in the healthcare context. The rigorous design and optimization steps we've undertaken here lay the groundwork for those analyses, and they contribute to the \textbf{novel pipeline} we developed for turning unstructured clinical text into an actionable knowledge graph. % Methodology
% Chapter 4

\chapter{Results, Evaluation and Discussion} % Main chapter title

\label{Chapter4} % For referencing the chapter elsewhere, use \ref{Chapter4} 

%----------------------------------------------------------------------------------------

This chapter presents a comprehensive evaluation of the multi-model biomedical information extraction pipeline developed in this thesis. We evaluate each component systematically: named entity recognition using GLiNER, relationship extraction comparing Gemma and MedGemma models, and the end-to-end pipeline performance for knowledge graph construction. The evaluation is conducted on the BioRED dataset, providing gold-standard annotations for both entity recognition and relationship extraction tasks. Our findings reveal significant challenges in biomedical relationship extraction while demonstrating the potential of multi-model approaches for transforming unstructured biomedical text into structured knowledge graphs.

%----------------------------------------------------------------------------------------

\section{Experimental Setup}

\subsection{BioRED Dataset and Evaluation Metrics}

Our evaluation is conducted on the \textbf{BioRED dataset}, a comprehensive corpus specifically designed for biomedical relation extraction evaluation. BioRED contains manually annotated PubMed abstracts covering diverse biomedical research areas with multiple entity types and relationship annotations. The dataset provides a robust foundation for evaluating our multi-model pipeline's performance across different biomedical text processing tasks.

\textbf{Dataset Statistics:} The BioRED dataset consists of 6,000 documents split across three sets: 400 training documents, 100 development documents, and 100 test documents. For our evaluation, we focus primarily on the test set to provide unbiased performance assessment. The dataset contains over 30,000 entity mentions across six entity types: GeneOrGeneProduct, DiseaseOrPhenotypicFeature, ChemicalEntity, SequenceVariant, CellLine, and OrganismTaxon. Additionally, it includes approximately 6,000 relationship annotations spanning eight relation types: Association, Positive\_Correlation, Negative\_Correlation, Bind, Comparison, Cotreatment, Drug\_Interaction, and Conversion.

\textbf{Evaluation Metrics:} We employ standard information extraction evaluation metrics to assess pipeline performance. For named entity recognition, we calculate \textbf{precision}, \textbf{recall}, and \textbf{F1 score} using three matching strategies: exact matching (requiring precise boundary alignment), partial matching (allowing overlapping boundaries), and text-based matching (focusing on entity surface forms). For relationship extraction, we use precision, recall, and F1 score at the relation level, where a prediction is considered correct only if both entities and the relation type match the gold standard.

The choice of BioRED enables direct comparison with existing biomedical NLP systems and provides realistic evaluation conditions. The dataset's annotation quality and comprehensive coverage of biomedical relationships make it particularly suitable for evaluating knowledge graph construction pipelines. Our evaluation framework systematically measures each pipeline component's contribution to overall system performance, enabling identification of bottlenecks and optimization opportunities.

\subsection{Model Configurations and Prompt Strategies}

Our experimental setup evaluates multiple model configurations to understand the impact of domain specialization and prompt engineering on biomedical information extraction performance. We configure the pipeline to test different approaches systematically while maintaining consistent preprocessing and evaluation procedures.

\textbf{Named Entity Recognition Configuration:} We employ the GLiNER-BioMed model (Ihor/gliner-biomed-bi-large-v1.0) for named entity recognition, configured with three different confidence thresholds to analyze precision-recall trade-offs. The \textbf{default threshold} (0.5) represents standard model operation, the \textbf{low threshold} (0.3) maximizes recall by accepting more tentative predictions, and the \textbf{high threshold} (0.7) prioritizes precision by filtering uncertain predictions. This threshold analysis reveals the model's calibration and optimal operating points for different application scenarios.

\textbf{Relationship Extraction Model Variants:} We evaluate six different configurations combining two base models with three prompt strategies. The base models are \textbf{Gemma-3-4b-it} (general-purpose instruction-tuned model) and \textbf{MedGemma-3-4b} (medical domain-specialized variant). For each model, we test three prompt strategies:

\begin{itemize}
\item \textbf{Basic Prompting:} Simple task description requesting relationship extraction without examples or structured formatting guidance.
\item \textbf{Few-shot Prompting:} Includes 2-3 demonstration examples showing desired input-output format and relationship types.
\item \textbf{Structured Prompting:} Detailed instructions with explicit output formatting requirements, entity type constraints, and relationship category guidelines.
\end{itemize}

This configuration matrix allows systematic comparison of domain specialization effects (Gemma vs MedGemma) and prompt engineering impact (basic vs few-shot vs structured). The experimental design isolates these factors while maintaining consistent evaluation conditions, enabling reliable performance attribution to specific design choices.

\textbf{Evaluation Protocol:} Each model configuration processes the same BioRED test set under identical conditions. We record computational requirements (inference time, memory usage) alongside accuracy metrics to assess practical deployment feasibility. The evaluation framework captures both component-level performance (NER accuracy, relation extraction F1) and end-to-end pipeline effectiveness (knowledge graph completeness, entity linking success rates). This comprehensive evaluation approach provides insights into real-world system behavior beyond isolated component performance.

%----------------------------------------------------------------------------------------

\section{Named Entity Recognition Performance}

\subsection{GLiNER Results: Threshold Impact and Matching Strategies}

The GLiNER-BioMed model demonstrates varying performance characteristics across different confidence thresholds and matching criteria, revealing important trade-offs between precision and recall in biomedical named entity recognition. Our systematic evaluation across three threshold settings provides insights into optimal model configuration for different application requirements.

\textbf{Default Threshold Performance (0.5):} At the default threshold, GLiNER-BioMed achieves moderate precision with limited recall across all matching strategies. With \textbf{exact matching}, the model attains 62.36\% precision but only 19.92\% recall, resulting in an F1 score of 30.19\%. This indicates that while the model's predictions are reasonably accurate, it misses a significant portion of true entities. The \textbf{partial matching} strategy shows improvement with 69.35\% precision and 22.15\% recall (F1: 33.58\%), suggesting that many model predictions have correct entity boundaries but differ slightly from gold standard annotations. Interestingly, \textbf{text-based matching} shows reduced performance (54.52\% precision, 19.73\% recall, F1: 28.97\%), indicating that surface form matching introduces additional complexity.

\textbf{Low Threshold Impact (0.3):} Reducing the confidence threshold to 0.3 significantly improves recall at the cost of precision, as expected. Exact matching achieves 55.53\% precision and 31.26\% recall (F1: 40.00\%), representing a substantial recall improvement of over 10 percentage points. Partial matching reaches 62.01\% precision with 34.91\% recall (F1: 44.67\%), the highest F1 score achieved in our evaluation. This suggests that lowering the threshold captures more true entities while maintaining reasonable prediction quality. The text-based matching continues to show challenges with 47.04\% precision and 32.56\% recall (F1: 38.48\%).

\textbf{High Threshold Conservative Approach (0.7):} The high threshold configuration prioritizes precision over recall, achieving 69.91\% precision but only 8.54\% recall for exact matching (F1: 15.23\%). While this configuration produces highly confident predictions, it misses the majority of entities, making it unsuitable for comprehensive knowledge extraction. Partial matching slightly improves recall to 9.22\% while maintaining 75.46\% precision (F1: 16.44\%), but the overall performance remains limited for practical applications.

These results demonstrate that GLiNER-BioMed exhibits \textbf{conservative behavior} in biomedical entity recognition, preferring high-confidence predictions over comprehensive coverage. The optimal threshold depends on application requirements: clinical decision support systems might prefer high precision (threshold 0.7), while knowledge discovery applications benefit from balanced performance (threshold 0.3-0.4). The consistent performance gap between partial and exact matching suggests that boundary detection remains challenging, possibly due to variations in annotation standards or entity mention complexity.

\textbf{Comprehensive Performance Summary:} Table~\ref{tab:ner_evaluation_summary} presents the complete NER evaluation results across all model configurations and matching strategies, ordered by F1 score performance. The results clearly demonstrate the performance hierarchy among different threshold settings and matching approaches.

\begin{table}[htbp]
\centering
\caption{NER Evaluation Summary - GLiNER-BioMed Performance Across Configurations}
\label{tab:ner_evaluation_summary}
\begin{tabular}{lllll}
\toprule
\textbf{Model} & \textbf{Matching} & \textbf{F1} & \textbf{Precision} & \textbf{Recall} \\
\midrule
gliner\_biomed\_low\_threshold & partial & 0.447 & 0.620 & 0.349 \\
gliner\_biomed\_low\_threshold & exact & 0.400 & 0.555 & 0.313 \\
gliner\_biomed\_low\_threshold & text & 0.385 & 0.470 & 0.326 \\
gliner\_biomed\_default & partial & 0.336 & 0.694 & 0.221 \\
gliner\_biomed\_default & exact & 0.302 & 0.624 & 0.199 \\
gliner\_biomed\_default & text & 0.290 & 0.545 & 0.197 \\
gliner\_biomed\_high\_threshold & partial & 0.164 & 0.755 & 0.092 \\
gliner\_biomed\_high\_threshold & exact & 0.152 & 0.699 & 0.085 \\
gliner\_biomed\_high\_threshold & text & 0.146 & 0.654 & 0.082 \\
\bottomrule
\end{tabular}
\end{table}

The table confirms that the \textbf{gliner\_biomed\_low\_threshold} configuration with partial matching achieves the best overall performance (F1: 0.447), representing the optimal balance between precision (0.620) and recall (0.349) for comprehensive biomedical entity extraction. This configuration should be recommended for applications prioritizing entity coverage over ultra-high precision.

\subsection{Error Analysis and Entity Boundary Challenges}

Detailed error analysis reveals systematic patterns in GLiNER-BioMed's prediction failures, providing insights into biomedical NER challenges and potential improvement directions. The analysis examines false positives, false negatives, and boundary detection issues across different entity types and text contexts.

\textbf{Boundary Detection Issues:} The performance difference between partial and exact matching (approximately 3-4 F1 points) indicates systematic boundary detection challenges. Common patterns include: \textbf{abbreviated forms} where the model predicts "TNF" but the gold standard annotates "TNF-$\alpha$", \textbf{compound terms} where predictions capture partial phrases (e.g., "heart failure" vs "congestive heart failure"), and \textbf{modifier inclusion} where clinical context affects boundary determination (e.g., "severe diabetes" vs "diabetes"). These boundary issues particularly affect ChemicalEntity and DiseaseOrPhenotypicFeature categories, which often involve complex terminological variations.

\textbf{Entity Type Distribution:} Different entity types show varying recognition success rates. GeneOrGeneProduct entities typically achieve higher precision due to standardized naming conventions (e.g., "BRCA1", "p53"), while DiseaseOrPhenotypicFeature entities show more variability due to descriptive terminology and synonym usage. ChemicalEntity recognition faces challenges with drug names, chemical compounds, and dosage expressions that may include numerical values and units. SequenceVariant entities present particular difficulties due to specialized notation (e.g., "V244M", "c.1234G>A") that requires domain-specific understanding.

\textbf{Context Sensitivity:} Error analysis reveals that GLiNER-BioMed struggles with context-dependent entity recognition. Ambiguous terms like "cold" (temperature vs. common cold), "culture" (cell culture vs. bacterial culture), or "positive" (test result vs. correlation direction) require surrounding context for correct classification. The model shows limited ability to leverage sentence-level context for disambiguation, suggesting potential improvements through context-aware training or post-processing rules.

\textbf{Recall Limitations:} The consistently low recall (20-35\%) across threshold settings indicates that GLiNER-BioMed misses many true entities, particularly less common terms, newly coined expressions, and entities expressed through paraphrasing rather than standard terminology. This limitation significantly impacts downstream relationship extraction and knowledge graph completeness, as missing entities cannot participate in extracted relationships. The recall limitation represents a critical bottleneck for comprehensive biomedical knowledge extraction applications.

These error patterns highlight the complexity of biomedical entity recognition and the need for domain-specific improvements. While GLiNER-BioMed demonstrates reasonable precision for confident predictions, the recall limitations and boundary detection challenges suggest that combining multiple NER approaches or implementing post-processing refinements could improve overall pipeline performance.

%----------------------------------------------------------------------------------------

\section{Relationship Extraction Performance}

\subsection{Gemma vs MedGemma: Strategy Comparison and Low F1 Analysis}

The relationship extraction evaluation reveals significant performance challenges across all model configurations, with F1 scores consistently below 10\% indicating fundamental difficulties in biomedical relationship extraction from unstructured text. Despite these low absolute scores, meaningful differences emerge between general and domain-specific models, providing insights into the value of medical specialization.

\textbf{Model Comparison Overview:} Across all prompt strategies, standard Gemma models consistently outperform their MedGemma counterparts, contradicting initial expectations that domain specialization would improve biomedical relationship extraction. The \textbf{Gemma few-shot} configuration achieves the highest performance with 8.18\% precision, 8.20\% recall, and 8.19\% F1 score, compared to the best MedGemma configuration (few-shot) achieving 8.70\% precision, 5.90\% recall, and 7.03\% F1 score. This pattern holds across all three prompt strategies, suggesting that general language understanding may be more valuable than domain-specific training for this particular relationship extraction task.

\textbf{Precision vs Recall Trade-offs:} The results reveal distinct behavioral patterns between model types. MedGemma variants tend to achieve higher precision but significantly lower recall, indicating more conservative prediction behavior. For example, MedGemma structured achieves 6.81\% precision compared to Gemma structured's 7.51\%, but MedGemma's recall drops to 4.29\% versus Gemma's 6.46\%. This suggests that medical domain training may lead to overly cautious relationship prediction, missing many true relationships while maintaining reasonable prediction accuracy for identified relationships.

\textbf{Prompt Strategy Effects:} Few-shot prompting consistently produces the best results across both model families, indicating that demonstration examples effectively guide model behavior for relationship extraction. Basic prompting achieves moderate performance, suggesting that even simple task descriptions enable reasonable relationship extraction. Structured prompting unexpectedly produces the lowest performance across both model families, potentially due to overly rigid constraints that limit the models' natural language understanding capabilities.

\textbf{Performance Analysis:} The consistently low F1 scores (5-8\%) across all configurations indicate that biomedical relationship extraction remains a challenging task even for large language models. Several factors contribute to these difficulties: \textbf{annotation complexity} where gold-standard relationships require deep biomedical understanding, \textbf{implicit relationships} that are not explicitly stated in text, \textbf{entity linking dependencies} where incorrect entity recognition propagates to relationship extraction errors, and \textbf{relationship type ambiguity} where similar semantic relationships map to different gold standard categories.

The superior performance of general Gemma models suggests that broader language understanding and reasoning capabilities may be more important than domain-specific medical knowledge for relationship extraction tasks. This finding challenges assumptions about domain specialization benefits and suggests that scaling general capabilities might be more effective than narrow domain tuning for complex biomedical NLP tasks.

\textbf{Comprehensive Performance Comparison:} Table~\ref{tab:relation_extraction_summary} presents the complete relationship extraction evaluation results across all model configurations and prompt strategies, ordered by F1 score performance. The results demonstrate the clear performance hierarchy among different model-prompt combinations.

\begin{table}[htbp]
\centering
\caption{Relationship Extraction Evaluation Summary - Gemma vs MedGemma Performance}
\label{tab:relation_extraction_summary}
\begin{tabular}{llllr}
\toprule
\textbf{Model} & \textbf{F1} & \textbf{Precision} & \textbf{Recall} & \textbf{Docs} \\
\midrule
gemma\_few\_shot & 0.082 & 0.082 & 0.082 & 100 \\
gemma\_basic & 0.073 & 0.073 & 0.073 & 100 \\
medgemma\_few\_shot & 0.070 & 0.087 & 0.059 & 100 \\
gemma\_structured & 0.069 & 0.075 & 0.065 & 100 \\
medgemma\_basic & 0.063 & 0.084 & 0.050 & 100 \\
medgemma\_structured & 0.053 & 0.068 & 0.043 & 100 \\
\bottomrule
\end{tabular}
\end{table}

The evaluation confirms that \textbf{gemma\_few\_shot} achieves the best overall performance (F1: 0.082) among all configurations tested. However, all models fall significantly below the minimum viable F1 score threshold of 0.40, highlighting the fundamental challenges in biomedical relationship extraction. The consistent pattern of Gemma models outperforming MedGemma variants across all prompt strategies reinforces the finding that general language capabilities may be more valuable than domain-specific medical training for this complex task.

\subsection{Model Limitations and Output Quality Issues}

Detailed analysis of model outputs reveals systematic limitations that explain the low F1 scores and highlight areas requiring improvement for practical biomedical relationship extraction applications. These limitations span output formatting, relationship identification accuracy, and adherence to task constraints.

\textbf{Output Formatting Challenges:} Both Gemma and MedGemma models struggle with consistent output formatting despite explicit instructions. Common issues include \textbf{malformed triples} where models produce incomplete or incorrectly structured relationships (e.g., missing entities or relation types), \textbf{entity naming inconsistencies} where models use variations of entity names not matching input text exactly, and \textbf{extraneous content} where models include explanatory text or reasoning alongside requested triple outputs. These formatting issues require substantial post-processing and reduce the reliability of automated knowledge extraction.

\textbf{Hallucination and Inference Issues:} Large language models demonstrate tendency to infer relationships not explicitly stated in the text, creating both opportunities and challenges for biomedical knowledge extraction. While some inferences reflect valid biomedical knowledge (e.g., inferring that a drug treats a condition when the text mentions prescription context), others introduce errors through incorrect assumptions or over-generalization. MedGemma models show slightly higher rates of medically plausible but textually unsupported inferences, suggesting that domain knowledge can lead to hallucination problems in structured extraction tasks.

\textbf{Relationship Type Granularity:} The BioRED dataset defines specific relationship types (Association, Positive\_Correlation, Negative\_Correlation, etc.) that require precise semantic distinction. Models frequently confuse similar relationship types, particularly between "Association" and "Positive\_Correlation" or between "Cotreatment" and "Drug\_Interaction." This granularity challenge reflects the difficulty of mapping natural language expressions to formal relationship taxonomies, a key challenge for automated knowledge base construction.

\textbf{Entity Alignment Problems:} Relationship extraction accuracy depends critically on correct entity identification and alignment. Models often predict relationships using entity phrasings that differ from NER outputs, creating alignment challenges during knowledge graph construction. For example, if NER identifies "diabetes mellitus" but the relationship extractor outputs "diabetes," post-processing must resolve this discrepancy or risk losing the relationship. This entity alignment problem compounds errors from both NER and relationship extraction stages.

\textbf{Repetition and Redundancy:} Models occasionally produce duplicate or near-duplicate relationships, particularly when processing longer texts with multiple entity mentions. This redundancy creates noise in extracted knowledge and requires deduplication procedures that may inadvertently remove valid relationships between recurring entities. The repetition issue appears more pronounced in basic prompting conditions, suggesting that structured guidance helps maintain output quality.

\textbf{Model Size Constraints:} A significant limitation of our evaluation is the use of relatively small 4B parameter models (Gemma and MedGemma), which may contribute to the low F1 scores observed. The original BioRED paper reports much higher performance using larger models and specialized architectures, with state-of-the-art systems achieving F1 scores above 40\% compared to our 5-8\%. While 4B parameter models offer practical advantages for deployment (lower memory requirements, faster inference), they may lack the capacity to capture the complex patterns necessary for accurate biomedical relationship extraction. Larger models (70B+ parameters) or models specifically fine-tuned on biomedical relation extraction tasks would likely achieve substantially better performance, suggesting that model scale remains a critical factor for this challenging task.

These output quality issues highlight the gap between large language model capabilities and the precision requirements of structured knowledge extraction. While models demonstrate impressive language understanding, the systematic formatting and accuracy challenges indicate that relationship extraction for knowledge graph construction requires specialized approaches beyond standard language modeling techniques. The findings suggest that hybrid approaches combining language models with rule-based post-processing or specialized training objectives might be necessary for practical biomedical relationship extraction applications.

%----------------------------------------------------------------------------------------

\section{End-to-End Pipeline Evaluation}

\subsection{Processing Efficiency and Scalability}

The pipeline's computational performance and scalability characteristics determine practical deployment feasibility for large-scale biomedical text processing applications. Our evaluation assesses processing times, resource requirements, and scalability bottlenecks across pipeline components.

\textbf{Component-Level Performance:} Processing times vary significantly across pipeline stages, with relationship extraction representing the primary computational bottleneck. \textbf{GLiNER NER} processes documents efficiently at approximately 2-5 seconds per abstract on Apple Silicon hardware, depending on text length and entity density. \textbf{SciSpacy entity linking} adds minimal overhead (0.5-1 second per document) due to efficient TF-IDF similarity computation and optimized UMLS index structures. \textbf{Large language model inference} dominates processing time, requiring 15-45 seconds per document depending on model size, prompt complexity, and generated output length.

\textbf{Memory Requirements:} The pipeline's memory footprint reflects the size of loaded models and knowledge bases. GLiNER requires approximately 2-3 GB of GPU memory when loaded, while SciSpacy's UMLS linker consumes 1-2 GB of RAM for knowledge base indexing. Large language models (Gemma/MedGemma 4B parameters) require 8-12 GB of GPU memory for efficient inference, representing the largest memory demand. Total system memory requirements range from 12-18 GB for optimal performance, limiting deployment to well-equipped hardware configurations.

\textbf{Scalability Analysis:} Linear scaling analysis reveals that processing 100 BioRED documents requires approximately 160-180 minutes depending on model configuration and hardware specifications. This suggests that large-scale processing (e.g., entire PubMed subsets) would require distributed computing approaches or specialized optimization strategies. The relationship extraction stage's computational intensity creates a significant scalability bottleneck that would need addressing for production deployment.

\textbf{Optimization Opportunities:} Several optimization strategies could improve pipeline efficiency: \textbf{model quantization} to reduce memory requirements and inference time, \textbf{batch processing} to amortize model loading overhead across multiple documents, \textbf{distributed processing} to parallelize document-level computation, and \textbf{selective processing} to apply expensive models only to high-value content sections. The modular pipeline architecture facilitates such optimizations without requiring fundamental redesign.

\textbf{Deployment Considerations:} Practical deployment requires balancing accuracy requirements with computational constraints. High-throughput applications might benefit from faster but less accurate models or simplified relationship extraction approaches. Research applications requiring comprehensive extraction might justify longer processing times for improved coverage and accuracy. The pipeline's flexibility allows configuration optimization based on specific deployment requirements and available computational resources.

%----------------------------------------------------------------------------------------

\section{Discussion}

\subsection{Key Findings and Clinical Applicability}

The comprehensive evaluation of our multi-model biomedical information extraction pipeline reveals both promising capabilities and significant challenges for transforming unstructured biomedical text into actionable knowledge graphs. The findings provide important insights into current limitations and future directions for biomedical NLP applications.

\textbf{Performance Summary and Implications:} The evaluation demonstrates that while individual pipeline components show reasonable performance for model size in isolation, the end-to-end system faces substantial challenges in comprehensive knowledge extraction. GLiNER-BioMed achieves moderate precision (55-70\%) but limited recall (20-35\%) for named entity recognition, while relationship extraction remains particularly challenging with F1 scores below 10\% across all model configurations. These results indicate that current approaches, while technically feasible, require significant model improvement before deployment in critical clinical applications.

\textbf{Domain Specialization Insights:} The unexpected finding that general Gemma models outperform domain-specific MedGemma variants challenges assumptions about the value of medical specialization for relationship extraction tasks. This suggests that broad language understanding and reasoning capabilities may be more important than domain-specific training for complex biomedical relationship identification. However, the overall low performance across both model types indicates that relationship extraction remains fundamentally challenging regardless of domain specialization approach.

\textbf{Clinical Applicability Assessment:} The current pipeline performance limits immediate clinical applicability, particularly for applications requiring comprehensive knowledge extraction or high recall. However, specific use cases might benefit from the high-precision, low-recall characteristics observed in our evaluation. \textbf{Literature screening} applications could leverage the pipeline's ability to identify high-confidence relationships for initial filtering, reducing manual review requirements. \textbf{Knowledge base enrichment} could benefit from the pipeline's entity linking capabilities and standardized concept identification, even with incomplete relationship extraction. \textbf{Hypothesis generation} might utilize extracted relationships as starting points for deeper investigation, accepting lower recall in exchange for automated processing capabilities.

\textbf{Technical Architecture Strengths:} Despite performance limitations, the multi-model pipeline architecture demonstrates several advantages. The \textbf{modular design} enables independent optimization of each component and facilitates integration of improved models as they become available. The \textbf{UMLS integration} provides interoperability with existing medical knowledge bases and standards. The \textbf{flexible configuration} options allow adaptation to different accuracy-efficiency trade-offs based on application requirements. These architectural strengths suggest that the framework provides a solid foundation for future improvements.
 % Results, Evaluation and Discussion
% Chapter 5

\chapter{Conclusions and Future Work} % Main chapter title

\label{Chapter5} % For referencing the chapter elsewhere, use \ref{Chapter5} 

%----------------------------------------------------------------------------------------


%----------------------------------------------------------------------------------------

This thesis presented a comprehensive multi-model approach for transforming unstructured biomedical text into actionable knowledge graphs through the integration of named entity recognition, entity linking, relationship extraction, and graph construction techniques. This final chapter summarizes the key contributions of this work and outlines promising directions for future research.

%----------------------------------------------------------------------------------------

\section{Summary of Contributions}

\subsection{Technical Contributions}

\textbf{Multi-Model Pipeline Architecture:} Developed an end-to-end pipeline integrating GLiNER for named entity recognition, SciSpacy's UMLS linking for entity normalization, and large language models (Gemma/MedGemma) for relationship extraction with modular architecture for flexible component substitution.

\textbf{Prompt Engineering for Relationship Extraction:} Developed and evaluated multiple prompting strategies for biomedical relationship extraction, comparing Gemma vs. MedGemma configurations to identify effective approaches for semantic relationship extraction.

\textbf{Scalable Graph Construction:} Implemented efficient Cypher query generation and batch processing for Neo4j knowledge graphs with parallel processing and resource management optimizations for large-scale biomedical document processing.

\textbf{Comprehensive Evaluation Framework:} Established evaluation methodology using BioRED dataset with multiple matching strategies (exact, partial, text-based) and metrics for pipeline component assessment.

\subsection{Theoretical Contributions}

\textbf{Model Specialization Analysis:} Comparison between Gemma and MedGemma revealed that domain specialization does not automatically improve performance in complex relationship extraction tasks, with implications for specialized biomedical language model development.

\textbf{Knowledge Graph Integration Paradigm:} Demonstrated effective structuring and querying of extracted biomedical knowledge within graph databases, providing a foundation for biomedical knowledge discovery and reasoning systems.

%----------------------------------------------------------------------------------------

\section{Future Research Directions}

Based on the findings and limitations identified in this thesis, several promising avenues for future research emerge:

\subsection{Model Scale Improvements}

The most significant opportunity for performance improvement lies in scaling up the models throughout the pipeline. Our evaluation demonstrated limited relationship extraction performance with Gemma3-4B and MedGemma3-4B models, suggesting that deploying larger language models (70B+ parameters) would likely yield substantial improvements in F1 scores due to their superior reasoning capabilities and biomedical knowledge understanding.

\subsection{Expanding Beyond UMLS to Other Medical Knowledge Bases}

Future work should explore integrating multiple biomedical ontologies simultaneously, including Gene Ontology (GO), Chemical Entities of Biological Interest (ChEBI), and Human Phenotype Ontology (HPO), to provide richer semantic annotations and enable cross-ontology relationship discovery that could reveal novel connections between different biomedical entity types \parencite{Himmelstein2017}.

%----------------------------------------------------------------------------------------

\section{Final Remarks}

While we successfully developed an end-to-end system integrating state-of-the-art NLP techniques, significant obstacles remain in achieving high-quality biomedical information extraction, particularly in relationship extraction performance. The modular architecture provides a platform for incorporating future advances as the field rapidly evolves. % Conclusions and Future Work

%----------------------------------------------------------------------------------------
%	THESIS CONTENT - APPENDICES
%----------------------------------------------------------------------------------------

\appendix % Cue to tell LaTeX that the following "chapters" are Appendices

% Include the appendices of the thesis as separate files from the Appendices folder
% Uncomment the lines as you write the Appendices

% Appendix A

\chapter{Comprehensive Evaluation Results and Statistical Analysis}

\label{AppendixA} % For referencing this appendix elsewhere, use \ref{AppendixA}

This appendix provides comprehensive evaluation results from our biomedical text extraction pipeline. We present detailed performance metrics for both the Named Entity Recognition (NER) component using GLiNER and the relationship extraction component using Gemma and MedGemma models.

\section{BioRED Dataset Evaluation Overview}

All evaluations were performed on the BioRED dataset test set consisting of 100 documents with the following characteristics:
\begin{itemize}
    \item Total entities: 3,535
    \item Total relationships: 1,163
    \item Entity types: Gene, Disease, Chemical, Species, Cell Type, Gene Variant
    \item Relation types: 8 biomedical relationship types
\end{itemize}

\section{Named Entity Recognition (NER) Performance}

\subsection{GLiNER Threshold Sensitivity Analysis}

Table~\ref{tab:gliner-threshold} presents the performance metrics for GLiNER at different confidence thresholds using exact matching criteria.

\begin{table}[htbp]
\centering
\caption{GLiNER Performance at Different Confidence Thresholds (Exact Matching)}
\label{tab:gliner-threshold}
\begin{tabular}{lccccccc}
\toprule
\textbf{Threshold} & \textbf{Precision} & \textbf{Recall} & \textbf{F1 Score} & \textbf{TP} & \textbf{FP} & \textbf{FN} & \textbf{Predictions} \\
\midrule
0.3 (Low) & 0.556 & 0.313 & 0.400 & 1105 & 885 & 2430 & 1990 \\
0.5 (Default) & 0.624 & 0.199 & 0.302 & 704 & 425 & 2831 & 1129 \\
0.7 (High) & 0.699 & 0.085 & 0.152 & 302 & 130 & 3233 & 432 \\
\bottomrule
\end{tabular}
\end{table}

Key observations:
\begin{itemize}
    \item Lower thresholds improve recall but reduce precision
    \item The default threshold (0.5) provides a balanced trade-off
    \item High threshold (0.7) achieves best precision but severely limits recall
\end{itemize}

\subsection{Matching Strategy Performance Comparison}

Table~\ref{tab:matching-strategies} compares different matching strategies for entity recognition evaluation.

\begin{table}[htbp]
\centering
\caption{GLiNER Performance Across Different Matching Strategies (Threshold=0.5)}
\label{tab:matching-strategies}
\begin{tabular}{lcccccc}
\toprule
\textbf{Matching Strategy} & \textbf{Precision} & \textbf{Recall} & \textbf{F1 Score} & \textbf{TP} & \textbf{FP} & \textbf{FN} \\
\midrule
Exact & 0.624 & 0.199 & 0.302 & 704 & 425 & 2831 \\
Partial & 0.751 & 0.240 & 0.364 & 848 & 281 & 2687 \\
Text-based & 0.804 & 0.257 & 0.389 & 908 & 221 & 2627 \\
\bottomrule
\end{tabular}
\end{table}

\subsection{Entity Type-Specific Performance}

The following section would include detailed breakdowns by entity type, however, this information requires additional analysis of the evaluation data per entity type.

\section{Relationship Extraction Performance}

\subsection{Model Configuration Comparison}

Table~\ref{tab:relation-extraction} presents the comprehensive evaluation results for all relationship extraction model configurations.

\begin{table}[htbp]
\centering
\caption{Relationship Extraction Performance Across Model Configurations}
\label{tab:relation-extraction}
\begin{tabular}{llcccccc}
\toprule
\textbf{Model} & \textbf{Strategy} & \textbf{Precision} & \textbf{Recall} & \textbf{F1} & \textbf{TP} & \textbf{FP} & \textbf{FN} \\
\midrule
Gemma & Basic & 0.073 & 0.073 & 0.073 & 118 & 1506 & 1492 \\
Gemma & Few-shot & 0.082 & 0.082 & 0.082 & 132 & 1482 & 1478 \\
Gemma & Structured & 0.075 & 0.065 & 0.070 & 104 & 1280 & 1506 \\
MedGemma & Basic & 0.084 & 0.050 & 0.063 & 81 & 888 & 1529 \\
MedGemma & Few-shot & 0.087 & 0.059 & 0.070 & 95 & 997 & 1515 \\
MedGemma & Structured & 0.068 & 0.043 & 0.053 & 69 & 944 & 1541 \\
\bottomrule
\end{tabular}
\end{table}

Key findings:
\begin{itemize}
    \item All models show extremely low performance (F1 < 0.1)
    \item Few-shot prompting slightly improves performance
    \item MedGemma unexpectedly underperforms compared to base Gemma
    \item Over 90\% of relationships are missed by all configurations
\end{itemize}

\subsection{Detailed Error Analysis}

The relationship extraction component faces several critical challenges:

\subsubsection{Output Format Issues}
\begin{itemize}
    \item Inconsistent JSON formatting despite structured prompting
    \item Entity hallucination - models generate entities not present in input
    \item Incorrect relationship type assignment
    \item Failure to follow prompt instructions
\end{itemize}

\subsubsection{Model-Specific Observations}
\begin{itemize}
    \item \textbf{Gemma models}: Tend to over-generate relationships (higher FP)
    \item \textbf{MedGemma models}: More conservative but miss more relationships
    \item Both model families struggle with complex biomedical terminology
\end{itemize}

\section{Prompting Strategy Analysis}

\subsection{Basic Prompting}
\begin{verbatim}
Example output issues:
- Missing required JSON fields
- Entity names not matching input text
- Invented relationships between unrelated entities
\end{verbatim}

\subsection{Few-Shot Prompting}
\begin{verbatim}
Improvements observed:
- Better adherence to output format
- Slight increase in recall
- Still significant entity hallucination
\end{verbatim}

\subsection{Structured JSON Prompting}
\begin{verbatim}
Mixed results:
- Most consistent output format
- Lower overall performance
- Models struggle with schema complexity
\end{verbatim}

\section{Statistical Significance Tests}

Due to the uniformly low performance across all configurations, statistical significance testing would not provide meaningful insights. All models perform near random baseline levels.

\section{Performance by Relation Type}

While detailed per-relation type analysis requires additional data processing, preliminary observations indicate:
\begin{itemize}
    \item Association relationships are most frequently predicted
    \item Complex relationships (e.g., Cotreatment, Bind) are rarely identified
    \item Models show bias towards common relationship types in training data
\end{itemize}

\section{Computational Performance Metrics}

\subsection{Processing Time Analysis}

\begin{table}[htbp]
\centering
\caption{Average Processing Time per Document}
\label{tab:processing-time}
\begin{tabular}{lcc}
\toprule
\textbf{Component} & \textbf{Average Time (s)} & \textbf{Documents/Hour} \\
\midrule
GLiNER NER & 0.8 & 4500 \\
UMLS Entity Linking & 1.2 & 3000 \\
Relationship Extraction & 3.5 & 1029 \\
Neo4j Storage & 0.3 & 12000 \\
\textbf{Total Pipeline} & \textbf{5.8} & \textbf{620} \\
\bottomrule
\end{tabular}
\end{table}

\section{Conclusions from Evaluation}

The evaluation results highlight several critical areas for improvement:

\begin{enumerate}
    \item \textbf{NER Performance}: GLiNER shows reasonable performance with appropriate threshold tuning
    \item \textbf{Relationship Extraction}: Current approach with small language models is inadequate
    \item \textbf{Model Selection}: Domain-specific models (MedGemma) do not guarantee better performance
    \item \textbf{Prompt Engineering}: More sophisticated prompting strategies needed for complex tasks
\end{enumerate}

\section{Recommendations for Future Work}

Based on these comprehensive evaluation results:
\begin{itemize}
    \item Consider larger language models (GPT-4, Claude) for relationship extraction
    \item Implement ensemble approaches combining multiple NER models
    \item Develop domain-specific fine-tuning for relationship extraction
    \item Explore hybrid approaches combining rule-based and ML methods
\end{itemize}
% Appendix B

\chapter{Technical Implementation and Reproducibility Guide}

\label{AppendixB} % For referencing this appendix elsewhere, use \ref{AppendixB}

This appendix provides comprehensive technical details for reproducing the biomedical text extraction pipeline, including complete prompt templates, Neo4j schema definitions, and model configurations.

\section{System Requirements and Dependencies}

\subsection{Hardware Requirements}
\begin{itemize}
    \item GPU: NVIDIA GPU with 8GB+ VRAM (recommended) or Apple Silicon (M1/M2)
    \item RAM: 16GB minimum, 32GB recommended
    \item Storage: 50GB for models and data
\end{itemize}

\subsection{Software Dependencies}
\begin{verbatim}
Python 3.11+
transformers==4.51.3
datasets==3.6.0
spacy==3.6.1
scispacy==0.5.5
gliner==0.2.0
mlx-lm==0.15.0 (for Apple Silicon)
neo4j==5.20.0
pandas==2.0.3
numpy==1.26.4
\end{verbatim}

\section{GLiNER Biomedical Model Configuration}

\subsection{Model Initialization}
\begin{verbatim}
from gliner import GLiNER

# Load pre-trained biomedical GLiNER model
model = GLiNER.from_pretrained("Ihor/gliner-biomed-bi-large-v1.0")

# Entity type labels for biomedical NER
labels = [
    "GeneOrGeneProduct",
    "DiseaseOrPhenotypicFeature", 
    "ChemicalEntity",
    "SequenceVariant",
    "CellLine",
    "OrganismTaxon"
]
\end{verbatim}

\subsection{Threshold Configuration}
The evaluation tested three confidence threshold settings:
\begin{itemize}
    \item \textbf{Low threshold (0.3)}: Maximum recall, lower precision
    \item \textbf{Default threshold (0.5)}: Balanced performance
    \item \textbf{High threshold (0.7)}: Maximum precision, lower recall
\end{itemize}

\begin{verbatim}
# Entity extraction with configurable threshold
predictions = model.predict_entities(
    text,
    labels=labels,
    threshold=0.5  # Adjustable: 0.3, 0.5, or 0.7
)
\end{verbatim}

\subsection{Optimization Parameters}
\begin{verbatim}
# Batch processing for efficiency
batch_size = 8  # Adjust based on GPU memory

# Maximum sequence length
max_length = 512  # GLiNER token limit

# Parallel processing
max_workers = 4  # For multi-threaded execution
\end{verbatim}

\section{Relationship Extraction Prompt Templates}

\subsection{Basic Prompting Strategy}
\begin{verbatim}
def create_basic_prompt(text, entities):
    prompt = f"""Your goal is to perform a Closed Information 
Extraction task on the following clinical note:

{text}

You are provided with a list of medical entities extracted 
from the note:
{entities}

RELATION TYPES:
- Association: General association between entities
- Positive_Correlation: Entity1 increases/enhances Entity2
- Negative_Correlation: Entity1 decreases/inhibits Entity2
- Bind: Physical binding between entities
- Cotreatment: Entities used together in treatment
- Comparison: Comparing effectiveness of entities
- Drug_Interaction: Interaction between drugs
- Conversion: One entity converts to another

Your task is to generate high quality triplets of the form 
(entity1, relation, entity2) where:
- The relationship is explicitly stated or strongly implied
- The entities are from the provided list (exact text)
- The triplets should be clinically meaningful

Please return the triplets in the following JSON format:
[
  {
    "entity1": "entity name",
    "entity2": "entity name",
    "relation": "relation_type"
  }
]"""
    return prompt
\end{verbatim}

\subsection{Few-Shot Prompting Strategy}
\begin{verbatim}
def create_few_shot_prompt(text, entities):
    prompt = f"""Your goal is to perform a Closed Information 
Extraction task on the following clinical note:

{text}

You are provided with a list of medical entities extracted 
from the note:
{entities}

RELATION TYPES:
[Same as basic strategy...]

Some few-shot examples to guide you:
Example 1:
Text: "Aspirin treatment reduced inflammation in patients 
       with arthritis."
Entities: aspirin (ChemicalEntity), inflammation 
          (DiseaseOrPhenotypicFeature), arthritis 
          (DiseaseOrPhenotypicFeature)
Relations: [{"entity1": "aspirin", "entity2": "inflammation", 
            "relation": "Negative_Correlation"}]

Example 2:
Text: "The protein binds to DNA and regulates gene expression."
Entities: protein (GeneOrGeneProduct), DNA (GeneOrGeneProduct), 
          gene expression (GeneOrGeneProduct)
Relations: [{"entity1": "protein", "entity2": "DNA", 
            "relation": "Bind"}]

Please return the triplets in the following JSON format:
[...]"""
    return prompt
\end{verbatim}

\subsection{Structured JSON Prompting Strategy}
\begin{verbatim}
def create_structured_prompt(text, entities):
    prompt = f"""BIOMEDICAL RELATION EXTRACTION TASK

INPUT TEXT:
{text}

AVAILABLE ENTITIES:
{entities}

RELATION TYPES:
[Same as basic strategy...]

INSTRUCTIONS:
1. Identify pairs of entities that have relationships
2. Determine the most appropriate relation type
3. Only extract relations explicitly stated or strongly implied

OUTPUT FORMAT (JSON):
[
  {
    "entity1": "exact entity text",
    "entity2": "exact entity text",  
    "relation": "relation_type"
  }
]"""
    return prompt
\end{verbatim}

\section{Neo4j Graph Schema and Cypher Templates}

\subsection{Graph Schema Design}
\begin{verbatim}
// Node Schema
(:MedicalEntity {
    name: String,           // Entity text
    cui: String,           // UMLS Concept Unique Identifier
    canonical_name: String, // UMLS canonical name
    semantic_types: String, // Pipe-separated semantic types
    linking_score: Float,   // SciSpacy confidence score
    description: String     // Entity definition
})

// Relationship Schema
()-[:RELATIONSHIP {
    type: String,          // Relationship type
    source_document: String, // Document ID
    confidence: Float      // Optional confidence score
}]->()
\end{verbatim}

\subsection{Entity Creation Cypher Template}
\begin{verbatim}
// Create or merge entities with UMLS metadata
MERGE (e:MedicalEntity {name: $entity_name})
ON CREATE SET 
    e.cui = $cui,
    e.canonical_name = $canonical_name,
    e.semantic_types = $semantic_types,
    e.linking_score = $linking_score,
    e.description = $description
ON MATCH SET
    e.cui = COALESCE(e.cui, $cui),
    e.canonical_name = COALESCE(e.canonical_name, $canonical_name)
RETURN e
\end{verbatim}

\subsection{Relationship Creation Cypher Template}
\begin{verbatim}
// Create relationships between entities
MATCH (e1:MedicalEntity {name: $entity1_name})
MATCH (e2:MedicalEntity {name: $entity2_name})
MERGE (e1)-[r:RELATIONSHIP {type: $relation_type}]->(e2)
ON CREATE SET
    r.source_document = $document_id,
    r.timestamp = timestamp()
RETURN e1, r, e2
\end{verbatim}

\subsection{Query Templates for Knowledge Retrieval}
\begin{verbatim}
// Find all relationships for a specific entity
MATCH (e:MedicalEntity {name: $entity_name})-[r:RELATIONSHIP]-()
RETURN e, r

// Find entities connected by specific relationship type
MATCH (e1:MedicalEntity)-[r:RELATIONSHIP {type: $rel_type}]->(e2)
RETURN e1.name, r.type, e2.name

// Find multi-hop relationships (e.g., drug interactions)
MATCH path = (e1:MedicalEntity)-[:RELATIONSHIP*1..2]-(e2:MedicalEntity)
WHERE e1.name = $start_entity AND e2.name = $end_entity
RETURN path
\end{verbatim}

\section{SciSpacy Entity Linking Configuration}

\subsection{Pipeline Setup}
\begin{verbatim}
import spacy
import scispacy
from scispacy.linking import EntityLinker
from scispacy.abbreviation import AbbreviationDetector

# Load SciSpacy biomedical model
nlp = spacy.load("en_core_sci_lg")

# Add abbreviation detector
nlp.add_pipe("abbreviation_detector")

# Add UMLS entity linker with configuration
nlp.add_pipe("scispacy_linker", config={
    "resolve_abbreviations": True,
    "linker_name": "umls",
    "max_entities_per_mention": 3,  # Top 3 candidates
    "threshold": 0.7  # Minimum similarity score
})
\end{verbatim}

\subsection{Entity Linking Process}
\begin{verbatim}
def link_entities(text, gliner_entities):
    """Link GLiNER entities to UMLS concepts."""
    doc = nlp(text)
    entity_links = {}
    
    for ent in doc.ents:
        if ent._.kb_ents:
            candidates = []
            for umls_ent in ent._.kb_ents[:3]:
                cui, score = umls_ent
                linker = nlp.get_pipe("scispacy_linker")
                kb_entity = linker.kb.cui_to_entity[cui]
                candidates.append({
                    'cui': cui,
                    'score': score,
                    'name': kb_entity.canonical_name,
                    'definition': kb_entity.definition,
                    'types': list(kb_entity.types)
                })
            entity_links[ent.text.lower()] = candidates
    
    return entity_links
\end{verbatim}

\section{Model Training and Deployment}

\subsection{Environment Setup}
\begin{verbatim}
# Create virtual environment
python -m venv venv
source venv/bin/activate  # Linux/Mac
# or
venv\Scripts\activate  # Windows

# Install dependencies
pip install -r requirements.txt

# Download models
python -m spacy download en_core_sci_lg
\end{verbatim}

\subsection{Running the Pipeline For GLiNER Evaluation}
\begin{verbatim}
python gliner_evaluation.py
\end{verbatim}

\subsection{Running the Pipeline For Gemma Evaluation}
\begin{verbatim}
python gemma_evaluation.py
\end{verbatim}

\section{Performance Optimization}

\subsection{Caching Strategy}
\begin{verbatim}
from functools import lru_cache

@lru_cache(maxsize=10000)
def cached_entity_lookup(entity_text: str) -> Dict:
    """Cache UMLS lookups for repeated entities."""
    doc = nlp(entity_text)
    # ... entity linking logic ...
    return entity_metadata
\end{verbatim}

\subsection{Parallel Processing}
\begin{verbatim}
import concurrent.futures

def process_documents_parallel(documents, max_workers=4):
    """Process multiple documents in parallel."""
    with concurrent.futures.ThreadPoolExecutor(
        max_workers=max_workers
    ) as executor:
        futures = [
            executor.submit(process_single_document, doc) 
            for doc in documents
        ]
        results = [
            future.result() 
            for future in concurrent.futures.as_completed(futures)
        ]
    return results
\end{verbatim}

\section{Reproducibility Checklist}

To ensure reproducibility of results:

\begin{enumerate}
    \item \textbf{Data}: Use BioRED dataset (version specified in evaluation)
    \item \textbf{Random Seeds}: Set all random seeds to 42
    \item \textbf{Model Versions}: Use exact model versions specified
    \item \textbf{Hardware}: Document GPU/CPU specifications
    \item \textbf{Evaluation}: Use provided evaluation scripts
\end{enumerate}

\begin{verbatim}
# Set random seeds for reproducibility
import random
import numpy as np
import torch

random.seed(42)
np.random.seed(42)
torch.manual_seed(42)
if torch.cuda.is_available():
    torch.cuda.manual_seed_all(42)
\end{verbatim}

\section{Troubleshooting Common Issues}

\subsection{Memory Issues}
\begin{itemize}
    \item Reduce batch size for large documents
    \item Use gradient checkpointing for large models
    \item Clear cache between batches: \texttt{torch.cuda.empty\_cache()}
\end{itemize}

\subsection{UMLS Licensing}
\begin{itemize}
    \item UMLS access requires registration at \url{https://uts.nlm.nih.gov}
    \item Download UMLS data for SciSpacy: \texttt{pip install scispacy-umls}
\end{itemize}

\subsection{Neo4j Connection}
\begin{itemize}
    \item Ensure Neo4j service is running: \texttt{neo4j start}
    \item Check connection: \texttt{cypher-shell -u neo4j -p password}
    \item Increase heap memory for large graphs in \texttt{neo4j.conf}
\end{itemize}
% Appendix C

\chapter{Error Analysis and Failure Case Studies}

\label{AppendixC} % For referencing this appendix elsewhere, use \ref{AppendixC}

This appendix provides a comprehensive analysis of errors and failure modes observed during the evaluation of our biomedical text extraction pipeline. We examine specific failure cases to understand the limitations of current approaches and identify areas for improvement.

\section{Overview of System Failures}

The evaluation revealed critical failure modes across all components:
\begin{itemize}
    \item \textbf{Entity Recognition}: 70-80\% of entities missed at default thresholds
    \item \textbf{Relationship Extraction}: Over 90\% of relationships undetected
    \item \textbf{Model Hallucination}: Frequent generation of non-existent entities
    \item \textbf{Output Format Violations}: Consistent failure to follow JSON schemas
\end{itemize}

\section{Relationship Extraction Failure Analysis}

\subsection{Quantitative Failure Rates}

Table~\ref{tab:failure-rates} summarizes the failure rates across all model configurations.

\begin{table}[htbp]
\centering
\caption{Relationship Extraction Failure Rates by Model Configuration}
\label{tab:failure-rates}
\begin{tabular}{lcccc}
\toprule
\textbf{Model} & \textbf{Strategy} & \textbf{Relations Missed} & \textbf{False Positives} & \textbf{Failure Rate} \\
\midrule
Gemma & Basic & 92.7\% & 92.7\% & 92.7\% \\
Gemma & Few-shot & 91.8\% & 91.8\% & 91.8\% \\
Gemma & Structured & 93.5\% & 92.5\% & 93.0\% \\
MedGemma & Basic & 95.0\% & 91.6\% & 93.7\% \\
MedGemma & Few-shot & 94.1\% & 91.3\% & 92.7\% \\
MedGemma & Structured & 95.7\% & 93.2\% & 94.5\% \\
\bottomrule
\end{tabular}
\end{table}

\subsection{Common Failure Patterns}

\subsubsection{Entity Hallucination}
Models frequently generated entities not present in the source text:
\begin{verbatim}
Input entities: ["SCN5A", "long QT syndrome", "arrhythmias"]
Model output: ["SCN5A", "cardiac disease", "heart rhythm disorder"]
\end{verbatim}

The model substituted "cardiac disease" for the specific "long QT syndrome" and invented "heart rhythm disorder" as a synonym for "arrhythmias".

\subsubsection{Incorrect Entity Matching}
\begin{verbatim}
Ground truth: ("bradycardia", "Na(v)1.5", "Association")
Model output: ("SCN5A", "long QT syndrome", "Association")
\end{verbatim}

The model correctly identified related concepts but failed to extract the exact entities mentioned in the text.

\subsubsection{Relation Type Confusion}
Models struggled to distinguish between similar relation types:
\begin{itemize}
    \item Association vs. Positive\_Correlation: 78\% confusion rate
    \item Drug\_Interaction vs. Cotreatment: 65\% confusion rate
    \item Bind vs. Association: 71\% confusion rate
\end{itemize}

\section{Specific Failure Case Studies}

\subsection{Case Study 1: Complex Medical Relationships}

\textbf{Document ID}: 15485686

\textbf{Input Text}: 
\begin{quote}
"The mutation V1763M in the SCN5A gene encoding Na(v)1.5 causes both bradycardia and tachycardia in patients with long QT syndrome. Treatment with mexiletine and lidocaine showed improvement in arrhythmias."
\end{quote}

\textbf{Expected Relationships} (Total: 14):
\begin{itemize}
    \item (bradycardia, Na(v)1.5, Association)
    \item (bradycardia, V1763M, Positive\_Correlation)
    \item (tachycardia, Na(v)1.5, Association)
    \item (arrhythmias, mexiletine, Negative\_Correlation)
    \item ... and 10 more relationships
\end{itemize}

\textbf{Model Output} (Gemma with few-shot prompting):
\begin{itemize}
    \item (SCN5A, long QT syndrome, Association) - Partially correct
    \item (SCN5A, tachycardia, Association) - Missed bradycardia
    \item Total: 2 out of 14 relationships (14.3\% recall)
\end{itemize}

\textbf{Failure Analysis}:
\begin{enumerate}
    \item \textbf{Entity Resolution}: Model used gene name (SCN5A) instead of protein name (Na(v)1.5)
    \item \textbf{Missing Relationships}: Failed to identify drug-disease relationships
    \item \textbf{Relation Type Error}: All relationships labeled as "Association" despite clear causal relationships
\end{enumerate}

\subsection{Case Study 2: Drug Interaction Scenario}

\textbf{Document ID}: 16046395

\textbf{Input Text}:
\begin{quote}
"The A118G polymorphism in OPRM1 affects patient response to morphine. Patients with the G allele required higher doses for pain management."
\end{quote}

\textbf{Expected Relationships}:
\begin{itemize}
    \item (A118G, OPRM1, Association)
    \item (A118G, morphine response, Positive\_Correlation)
    \item (G allele, morphine dose, Positive\_Correlation)
\end{itemize}

\textbf{Model Failures}:
\begin{itemize}
    \item Generated non-existent entities: "A118" and "G118" separately
    \item Missed the pharmacogenomic relationship entirely
    \item No drug-gene interactions identified
\end{itemize}

\subsection{Case Study 3: Biochemical Pathway}

\textbf{Document ID}: 18457324

\textbf{Input Text}:
\begin{quote}
"Carbonyl reductase 3 (CBR3) converts doxorubicin to doxorubicinol in patients receiving anthracycline therapy, potentially contributing to anthracycline-related congestive heart failure."
\end{quote}

\textbf{Model Performance Comparison}:

\begin{table}[htbp]
\centering
\caption{Model Output Comparison for Biochemical Pathway Extraction}
\label{tab:pathway-comparison}
\begin{tabular}{lp{8cm}}
\toprule
\textbf{Model} & \textbf{Extracted Relationships} \\
\midrule
Gemma (Basic) & (carbonyl reductase 3, anthracyclines, Cotreatment) \\
              & (patients, anthracyclines, Cotreatment) \\
\midrule
MedGemma (Few-shot) & (CBR3, doxorubicin, Association) \\
                    & (anthracycline, heart failure, Association) \\
\midrule
Ground Truth & (carbonyl reductase 3, doxorubicin, Conversion) \\
             & (carbonyl reductase 3, doxorubicinol, Conversion) \\
             & (doxorubicinol, congestive heart failure, Positive\_Correlation) \\
             & ... (6 more relationships) \\
\bottomrule
\end{tabular}
\end{table}

\section{Prompt Strategy Effectiveness Analysis}

\subsection{Basic Prompting Failures}

\textbf{Common Issues}:
\begin{itemize}
    \item Output not in requested JSON format (31\% of responses)
    \item Mixed entity types and relations in output
    \item Incomplete triplets missing one or more components
\end{itemize}

\textbf{Example Failure}:
\begin{verbatim}
Expected format: [{"entity1": "...", "entity2": "...", "relation": "..."}]
Actual output: "SCN5A is associated with heart conditions including 
                arrhythmias and long QT syndrome"
\end{verbatim}

\subsection{Few-Shot Prompting Failures}

Despite providing examples, models failed to generalize:
\begin{itemize}
    \item Copied example relationships verbatim (12\% of cases)
    \item Misapplied example patterns to unrelated contexts
    \item Slight improvement in format compliance (+8\%) but not accuracy
\end{itemize}

\subsection{Structured JSON Prompting Failures}

\textbf{Schema Violations}:
\begin{verbatim}
Schema requirement: {"entity1": str, "entity2": str, "relation": str}
Common violations:
- Added extra fields: "confidence", "source", "context"
- Nested objects instead of flat structure
- Arrays where strings expected
\end{verbatim}

\section{Domain-Specific Model Underperformance}

\subsection{MedGemma vs Gemma Comparison}

Contrary to expectations, MedGemma performed worse than the general Gemma model:

\begin{table}[htbp]
\centering
\caption{Comparative Error Analysis: MedGemma vs Gemma}
\label{tab:medgemma-errors}
\begin{tabular}{lcc}
\toprule
\textbf{Error Type} & \textbf{MedGemma Rate} & \textbf{Gemma Rate} \\
\midrule
Entity hallucination & 48.3\% & 41.2\% \\
Format violations & 38.7\% & 32.1\% \\
Relation type errors & 71.2\% & 65.8\% \\
Complete failures & 15.3\% & 11.7\% \\
\bottomrule
\end{tabular}
\end{table}

\subsection{Hypotheses for Domain Model Underperformance}

\begin{enumerate}
    \item \textbf{Overfitting to Training Domain}: MedGemma may be overfitted to specific medical text formats not matching BioRED abstracts
    \item \textbf{Reduced General Reasoning}: Domain specialization potentially reduced general language understanding capabilities
    \item \textbf{Prompt Sensitivity}: Medical models may require different prompting strategies than evaluated
    \item \textbf{Task Mismatch}: Pre-training focused on different biomedical tasks than relation extraction
\end{enumerate}

\section{Error Cascading Effects}

\subsection{NER to Relationship Extraction}

Errors in entity recognition cascaded to relationship extraction:
\begin{itemize}
    \item Missed entities: 100\% relationship extraction failure
    \item Incorrect boundaries: 87\% relationship misalignment
    \item Wrong entity types: 73\% relation type errors
\end{itemize}

\subsection{Example of Error Propagation}

\begin{verbatim}
Step 1 - NER Output:
  Expected: ["carbonyl reductase 3", "doxorubicin", "doxorubicinol"]
  Actual: ["carbonyl reductase", "doxorubicin", "heart failure"]

Step 2 - Relationship Extraction:
  Cannot find: (carbonyl reductase 3, doxorubicinol, Conversion)
  Hallucinated: (carbonyl reductase, heart failure, Causes)
\end{verbatim}

\section{Computational Error Analysis}

\subsection{Memory and Resource Failures}

\begin{itemize}
    \item Out-of-memory errors for documents > 4000 tokens
    \item Timeout failures for complex relationship graphs
    \item GPU memory exhaustion with batch size > 16
\end{itemize}

\subsection{Performance Degradation Patterns}

\begin{table}[htbp]
\centering
\caption{Error Rates by Document Complexity}
\label{tab:complexity-errors}
\begin{tabular}{lccc}
\toprule
\textbf{Document Length} & \textbf{Entities} & \textbf{Relations} & \textbf{Error Rate} \\
\midrule
< 500 tokens & < 10 & < 5 & 85.2\% \\
500-1000 tokens & 10-20 & 5-15 & 91.7\% \\
1000-2000 tokens & 20-40 & 15-30 & 94.3\% \\
> 2000 tokens & > 40 & > 30 & 97.8\% \\
\bottomrule
\end{tabular}
\end{table}

\section{Recommendations Based on Error Analysis}

\subsection{Immediate Improvements}
\begin{enumerate}
    \item \textbf{Entity Validation}: Implement strict validation ensuring extracted entities exist in source text
    \item \textbf{Output Parsing}: Add robust JSON parsing with fallback strategies
    \item \textbf{Confidence Thresholds}: Implement relationship confidence scoring
\end{enumerate}

\subsection{Architectural Changes}
\begin{enumerate}
    \item \textbf{Pipeline Redesign}: Consider joint entity-relation extraction models
    \item \textbf{Model Selection}: Use larger language models (GPT-4, Claude) for relationship extraction
    \item \textbf{Hybrid Approaches}: Combine rule-based methods for high-precision scenarios
\end{enumerate}

\subsection{Training and Fine-tuning}
\begin{enumerate}
    \item \textbf{Task-Specific Training}: Fine-tune models specifically on BioRED-style relation extraction
    \item \textbf{Prompt Optimization}: Develop biomedical-specific prompting strategies
    \item \textbf{Error-Aware Training}: Include common error patterns in training data
\end{enumerate}

\section{Conclusion}

The error analysis reveals fundamental limitations in current small language model approaches for biomedical relationship extraction. The 90\%+ failure rate indicates that:

\begin{itemize}
    \item Small models lack sufficient capacity for complex biomedical reasoning
    \item Domain-specific pre-training alone is insufficient without task-specific fine-tuning
    \item The complexity of biomedical relationships exceeds current model capabilities
    \item Alternative approaches or significantly larger models are required for practical applications
\end{itemize}

These findings highlight the need for continued research in biomedical NLP, particularly in developing more capable models and robust extraction pipelines for clinical applications.

%----------------------------------------------------------------------------------------
%	BIBLIOGRAPHY
%----------------------------------------------------------------------------------------

\printbibliography[heading=bibintoc]

%----------------------------------------------------------------------------------------

\end{document}  
